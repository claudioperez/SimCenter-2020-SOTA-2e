%%%%%%%%%%%%%%%%%%%% author.tex %%%%%%%%%%%%%%%%%%%%%%%%%%%%%%%%%%%
%
% sample root file for your "contribution" to a contributed volume
%
% Use this file as a template for your own input.
%
%%%%%%%%%%%%%%%% Springer %%%%%%%%%%%%%%%%%%%%%%%%%%%%%%%%%%%%%%%%%


%% RECOMMENDED %%%%%%%%%%%%%%%%%%%%%%%%%%%%%%%%%%%%%%%%%%%%%%%%%%%
%\documentclass[graybox]{svmult}
%
%% choose options for [] as required from the list
%% in the Reference Guide
%
%\usepackage{mathptmx}       % selects Times Roman as basic font
%\usepackage{helvet}         % selects Helvetica as sans-serif font
%\usepackage{courier}        % selects Courier as typewriter font
%\usepackage{type1cm}        % activate if the above 3 fonts are
                             % not available on your system
%
%\usepackage{makeidx}         % allows index generation
%\usepackage{graphicx}        % standard LaTeX graphics tool
%                             % when including figure files
%\usepackage{multicol}        % used for the two-column index
%\usepackage[bottom]{footmisc}% places footnotes at page bottom
%
%% see the list of further useful packages
%% in the Reference Guide
%
%\makeindex             % used for the subject index
%                       % please use the style svind.ist with
%                       % your makeindex program
%
%%%%%%%%%%%%%%%%%%%%%%%%%%%%%%%%%%%%%%%%%%%%%%%%%%%%%%%%%%%%%%%%%%%%%%%%%%%%%%%%%%%%%%%%%%
%
%\begin{document}

\title{Earthquake---Soil Liquefaction}
% Use \titlerunning{Short Title} for an abbreviated version of
% your contribution title if the original one is too long
\author{
    \textbf{Chaofeng Wang}
    with contributions by Jonathan D. Bray,
    \newline
    along with review comments and suggestions by Pedro Arduino, Brady Cox, and Michael Gardner
}
\tocauthor{}
\authorrunning{Wang}
% Use \authorrunning{Short Title} for an abbreviated version of
% your contribution title if the original one is too long
%\institute{Name of First Author \at Name, Address of Institute, %\email{name@email.address}
%\and Name of Second Author \at Name, Address of Institute %\email{name@email.address}}
%
% Use the package "url.sty" to avoid
% problems with special characters
% used in your e-mail or web address
%
\maketitle

Soil liquefaction has caused much damage in recent earthquakes (e.g, \cite{cubrinovski2011geotechnical, cubrinovski2017liquefaction, bray2017new}). In fluid-saturated, geologically unconsolidated soil, liquefaction is a phenomenon caused by a sudden change in stress due to a rapid earthquake loading, during which the pore-water pressure increases and the effective stress reduces so that the soil loses much of its stiffness and strength and behaves like a liquid. The liquefied soil may lose its ability to support overlying structures or buried utilities. 

As summarized in the \citet{national2016state} report, the consequences of liquefaction may include vertically or laterally displaced ground, landslides, slumped embankments, foundation failures, and mixtures of soil and water erupting at the ground surface. In turn, these effects may lead to settlement, distortion, and the collapse of buildings; the disruption of roadways; the failure of earth-retaining structures; the cracking, sliding, and over-topping of dams, highway embankments, and other earth structures; the rupture or severing of sewer, water, fuel, and other lifeline infrastructure; the lateral displacement and shear failure of piles supporting bridges and waterfront structures; and the uplift of underground structures.

\section{Input and Output Data}

Input data for assessing the liquefaction hazard describe the earthquake intensity measures (IMs), the characteristics of geologic and technical site conditions, and the topography of the site.

\subsection{Input Data}
\label{subsec:eq_liquefaction_input}

\paragraph{Ground motion and intensity measures}
Mechanics-based numerical simulation of liquefaction requires ground motions as input data. Simplified empirical methods take earthquake moment magnitude and IMs as input data that include peak ground acceleration (PGA), spectral acceleration ($Sa(T)$), Arias intensity ($A_I$), etc. Which IM to use depends on the selected analytical models.


\paragraph{Site characteristics}
Site characteristics include geologic and geotechnical site conditions, water table, slope, and topography.

\subparagraph{Geologic characteristics}
Regional liquefaction assessment depends on geologic information. \citet{youd1987mapping}  proposed a method for classifying the liquefaction susceptibility of numerous geologic units. This requires the mapping of geologic units, which are generally characterized by their depositional environment, physical characteristics, and age. The depth to the groundwater table is also required.

\subparagraph{Geotechnical characteristics}
Geotechnical characterization of the site consists primarily of identifying soil properties, such as soil type, shear strength, density, fines content, water saturation, etc. These parameters can be estimated by \textit{in situ} tests, such as the cone penetration test (CPT), standard penetration test (SPT), and shear-wave velocity (${V_s}$) measurement, or laboratory testing of soil specimens.

\subparagraph{Topography}
Ground topographic parameters (e.g., ground slope, free face height, and the distance to a free face) define the boundary conditions of a site and are input data required for a liquefaction assessment. 


\subsection{Output Data}
\label{subsec:eq_liquefaction_output}
Depending on the modeling approach used, one or more of the following outputs can be produced:

\paragraph{Liquefaction indices at a site}
Based on input data collected at a site, various liquefaction indices can be calculated, including liquefaction potential index (LPI), liquefaction severity number (LSN), etc.

\paragraph{Induced ground displacement at a site}
Liquefaction in the soil can lead to the deformation of the ground surface, such as vertical settlement and lateral spreading.

\paragraph{Liquefaction maps}
Maps of liquefaction susceptibility or liquefaction-induced ground damage can be the outputs of a regional liquefaction assessment. 

\section{Modeling Approaches}
\label{sec:eq_liq_methods}

Analysis of liquefaction and its consequences is an active area of research and development in geotechnical engineering. Methods for estimating liquefaction triggering and its consequences vary. They fall into two categories: simplified methods and mechanics-based numerical methods.
% \newline

\paragraph{Simplified methods} 
In 1998, a consensus was reached within the geotechnical community on the use of an empirical stress-based approach for liquefaction triggering assessment called the ``simplified method,'' first developed by \citet{seed1971simplified}. This method is still commonly used in practice today \citep{youd2001liquefaction, national2016state}.

In a simplified method, a factor of safety (FS) against liquefaction triggering, defined as the ratio between the seismic loading required to trigger liquefaction (i.e., the liquefaction resistance) and the seismic loading expected from the earthquake (i.e., the seismic demand), is computed. Both the seismic demand and the liquefaction resistance are characterized as cyclic stress ratios, defined as the ratio of the equivalent cyclic shear stress to the initial vertical effective stress. The seismic demand is the earthquake-induced cyclic stress ratio (CSR), and the liquefaction resistance is the cyclic resistance ratio (CRR): that is, the cyclic stress ratio required to trigger liquefaction. 

\citet{seed1971simplified} proposed a simplified equation, based on Newton's second law, to compute a representative CSR for a given earthquake magnitude. This model was later revised by \citet{idriss1999update, cetin2004nonlinear, idriss2008soil, boulanger2014cpt}.

%NOTE: reword?
The most common approaches used in practice to compute CRR are based on geotechnical field data, e.g., CPT, SPT, and $V_s$. The most commonly used relationships to estimate CRR from a CPT profile are those developed by \citet{robertson1998evaluating, moss2006cpt, idriss2008soil, robertson2009interpretation}. The most commonly used relationships to estimate CRR from SPT blow count are those proposed by \citet{youd2001liquefaction, boulanger2014cpt, cetin2018sptbased}. The most commonly used relationships to estimate CRR from a $V_s$ profile were developed by \citet{andrus2000liquefaction, kayen2013shearwave}.

\paragraph{Mechanics-based numerical simulation} The development and validation of numerical analysis tools and procedures for estimating the effects of liquefaction on the built environment is identified as an overarching research need \citep{bray2017new}. Numerical analysis is critical for several reasons, including obtaining insights on field mechanisms that cannot be discerned empirically, providing a rational basis for developing or constraining practice-oriented engineering models, and providing a tool for evaluating complex structures with unique characteristics that are outside the range of empirical observations. 

Finite-element and finite-difference procedures are the most common procedures used in engineering practice. As pointed out in \citet{bray2017new}, there are major challenges in developing robust validated numerical analysis procedures for evaluating the effects of liquefaction on civil infrastructure systems due to the variety of multi-scale, multi-physics coupled nonlinear interactions that come to the forefront in different scenarios where analytical capabilities for liquefaction effects have not been validated. Currently, research or commercial software platforms have not incorporated the best available solution techniques/options for these challenging problems, such as the coupled, large-deformation analysis of strain-softening, localizations, cracking, and interfaces in two or three dimensions with complex constitutive models. 

\section{Research Gaps and Opportunities}
\label{sec:eq_liq_gaps}

Liquefaction risk analyses has focused on assessing the likelihood of triggering of liquefaction, resulting in maps of susceptibility. The consequences of liquefaction, i.e., the induced damage to the ground, is not sufficiently studied. Well established and universally applicable methods for quantifying liquefaction-related damages to the ground are still to be developed. HAZUS provides a method derived from engineering experiences for evaluating liquefaction-induced permanent ground deformation (PGD) given PGA and site-specific liquefaction susceptibility; however, the HAZUS method was developed based on observations of relatively old events.

In recent years, with more liquefaction cases observed \citep{cubrinovski2017liquefaction, bray2017new} and the creation of large liquefaction databases \citep{brandenberg2020nextgeneration}, new regression-based ground damage models are being developed \citep{khoshnevisan2015probabilistic, stewart2016peerngl} and tested for different regions \citep{chen2016probabilistic}. New insights into the consequent damage to overlaying structures and their fragility are also being developed \citep{bray20176th, fotopoulou2018vulnerability}. In regional simulations, large-scale assessment of liquefaction is needed. Traditionally, regional liquefaction can be coarsely assessed based on geological data \citep{holzer2006liquefaction}. In recent years, techniques such as random fields are proposed for regional-scale modeling of liquefaction hazard, which can account for spatial uncertainties while considering both geotechnical and geological information (\cite{zhu2017updated, wang2017spatial, wang2018hybrid}).



\section{Software and Systems}
\label{sec:eq_liq_tools}

Systems for liquefaction evaluation are divided into two categories according to the method used: simplified empirical methods and mechanics-based numerical methods.

\paragraph{Simplified methods}
Simplified methods have been developed for rapid engineering evaluations of site-specific liquefaction. To date, no open-source software is available. \citeprgm{LiqIT}, \citeprgm{Cliq}, \citeprgm{NovoLIQ}, and \citeprgm{Liquefy-Pro} are all Windows based. These tools provide a user interface that can let the user input the soil profiles and earthquake loading, and then visualize the liquefaction index for each soil layer.

\paragraph{Numerical methods}
For mechanics-based numerical methods, creating a constitutive model that captures the soil's behavior under cyclic loads is crucial. Commercial software such as \citeprgm{PLAXIS} and \citeprgm{FLAC} are widely used by the geotechnical community. Both of them are Windows based. OpenSees is the only open-source software identified for dealing with liquefaction. Several well-known liquefaction-capable constitutive models are: PM4Sand, PM4Silt, PDMY02, UBCSAND, and DAFALIAS-MANZARI. Except for UBCSAND, they are all available in OpenSees.

\subsection{Relevant SimCenter Tools}

The SimCenter develops both research and educational tools to facilitate performing site-specific analysis of soil response to earthquakes. The tools currently available focus on the one-dimensional propagation of ground shaking from the bedrock to the free surface.

\paragraph{QS3HARK} The Site-Specific Seismic Hazard Analysis and Research Kit with Uncertainty Quantification (\emph{QS3HARK}) performs site-specific analysis of ground shaking and liquefaction by simulating wave propagation through soil layers. The simulations use the finite element method as implemented in \citeprgm{OpenSees} to perform the calculations. Several advanced material models are available in QS3HARK (e.g., PM4Sand, PM4Silt, PDMY, PDMY02, PDMY03, ManzariDafalias, Borja-Amies) to support advanced site-response analysis. Uncertainties in both the soil properties and the bedrock ground motion inputs can be characterized and propagated throughout the simulations to arrive at a probabilistic description of the ground shaking, ground deformation, and the liquefied soil layers.

\paragraph{S3HARK} The Site-Specific Seismic Hazard Analysis and Research Kit (\emph{S3HARK}) is the educational version of QS3HARK that provides the same features and versatile set of material models but without uncertainty quantification. Removing UQ supports educational needs by streamlining the user interface and facilitating the setup of simulations.