
\begin{partbacktext}
\part{Hazard Characterization}
\label{part:hazard}

Characterization of natural hazards for engineering applications aims to quantify the severity of a natural hazard at a particular location or over a pre-defined region of interest. The time histories that provide a detailed description of natural-hazard effects---such as severe ground shaking or high-speed winds---are often summarized by so-called intensity measures (IMs) that represent their most important characteristics. Peak ground acceleration, permanent ground deformation, average one-minute wind speed, and peak inundation depth are a few examples of such IMs for various natural hazards. Using IMs facilitates the development of a stochastic hazard model that defines the hazard severity at the site(s) of interest using one more more random variables or random fields. The uncertainty in the hazard that is captured by these random entities shall be propagated through engineering simulation. 

High-fidelity approaches simulate the time-history of structural response using dynamic analyses (see Part \ref{part:response} for details). These simulations require a time-dependent load function that represents the hazard characterized by IMs. An acceleration time history of a ground motion is an example of such a load function, which is often used for seismic response estimation. These load functions are either selected from historical data (e.g., ground-motion records) or generated using a stochastic process (e.g., local wind inflow conditions for a CFD simulation). The procedures and best practices available for this task will be discussed for each natural hazard below.

Three types of natural hazards are examined in the following sections: earthquake, hurricane, and tsunami. They present several fundamentally different threats to the built environment, such as ground shaking and liquefaction under earthquakes, or wind and storm surge under hurricanes. Each chapter in this part discusses one of those threats.

\end{partbacktext}