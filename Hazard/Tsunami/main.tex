%%%%%%%%%%%%%%%%%%%% author.tex %%%%%%%%%%%%%%%%%%%%%%%%%%%%%%%%%%%
%
% sample root file for your "contribution" to a contributed volume
%
% Use this file as a template for your own input.
%
%%%%%%%%%%%%%%%% Springer %%%%%%%%%%%%%%%%%%%%%%%%%%%%%%%%%%%%%%%%%


\title{Tsunami---Inundation}
% Use \titlerunning{Short Title} for an abbreviated version of
% your contribution title if the original one is too long
\author{
    \textbf{Michael Motley}
    with contributions by Andrew Winter,
    \newline
    along with review comments and suggestions by Ajay B. Harish, Andrew Kennedy, and Rick Luettich
}
\tocauthor{}
\authorrunning{Motley}
% Use \authorrunning{Short Title} for an abbreviated version of
% your contribution title if the original one is too long
%\institute{Name of First Author \at Name, Address of Institute, %\email{name@email.address}
%\and Name of Second Author \at Name, Address of Institute %\email{name@email.address}}
%
% Use the package "url.sty" to avoid
% problems with special characters
% used in your e-mail or web address
%
\maketitle
\label{chapter:haz_tsunami}

The simulation tools used to model tsunami inundation can generally be categorized as either two-dimensional or three-dimensional models. Three-dimensional models solve Reynolds average Navier-Stokes (RANS) or Large Eddy Simulation (LES) equations numerically using the CFD techniques introduced in Chapter \ref{chapter:res_water}. Two-dimensional models often solve a different category of governing equations derived from the three-dimensional Navier-Stokes (NS) equation by integrating in the vertical dimension, e.g., shallow water equations and Boussinesq wave equations where the two dimensions can be characterized as latitude and longitude, with additional techniques available in some applications to consider variations in the vertical direction. Because such equations are computationally much easier to solve than the RANS and LES equations, they are broadly used in large-scale modeling of geophysical flows, e.g. tsunami, storm surge, and flooding. Two-dimensional models are used in those cases where computational efficiency is a factor. Examples include building an early warning system, which requires tsunami modeling to be done as quickly as possible after an earthquake, versus a probabilistic assessment that often requires thousands of computational simulations.

\section{Input and Output Data}
\label{sec:tsunami_io}

Both two-dimensional and three-dimensional models need boundaries for the simulation and boundary conditions as input. In two-dimensional models, buildings and bridges cannot be described directly.  Buildings, because they come into direct contact with the ground, are often incorporated as part of the ground topography and represented as an elevation field that varies in horizontal space, or by simply removing a volume of the water column in the shape of the building footprint from the computational domain. Bridges, on the other hand, are much more difficult to represent because of the three-dimensional complexities of the bridge geometry and the surrounding area.  To define the computational domain, a finite region in the horizontal plane must be specified, inside of which the flows are modeled. The boundary conditions on the boundary of this finite region must be specified, e.g., flows are allowed to flow out of the region freely on one side, and/or are reflected on the other. The topography (or bathymetry) data that describe the shape of the ground (or sea floor) must also be specified as input, although these are not interpreted as boundaries since the vertical dimension vanishes in a two-dimensional model, and the topography data are treated as field variables that directly affect other field variables (like velocities) in the solver.

Generally for three-dimensional models, flows must also be bounded in the vertical direction where the top boundary often represents the atmosphere and the bottom boundary represents the ground. Additionally, the simulation incorporates buildings and bridges into the simulation by subtracting volumes that describe the geometry of those structures from the domain. The surface of these volumes thus becomes boundaries of the simulation domain as well, and their boundary conditions must be specified. Thus, the geometry of the structures must be provided as input if one wants to model them as well.

Both types of models also need input of initial conditions: namely, the state of the fluids before the simulation starts. For instance, the initial conditions for some nearshore regions might have the fluid at rest at sea level, while somewhere far from shore, a large volume of water is placed above sea level to represent a tsunami wave. Some two-dimensional models will also incorporate seismically-induced sea floor displacement profiles or other submarine tsunamigenic phenomena to be used as initial conditions at the ocean surface above where they occur. Different initial conditions will result in different states later in time.

The output quantities from both two-dimensional and three-dimensional models include water surface and flow velocity; however, three-dimensional models are able to output quantities that vary in the vertical direction. The two-dimensional models generally do not depend on the vertical direction; therefore, its output quantities are generally not a function of positions in the vertical direction. Furthermore, the three-dimensional models can usually output more quantities of interest, e.g., water pressure, which can be integrated to obtain fluid loading on structures.

\section{Models and Software Systems}
\label{sec:tsunami_tools}

In general, tsunami simulation requires modeling at a wide range of spatial scales, including (from large- to small-scale) offshore propagation, beach run-up, inland inundation, and impact on individual structures.

For modeling that focuses on the large-scale phases, two-dimensional models are still the most prevalent choices for their simplicity and computational efficiency. Two major variants in this category are based on the shallow-water equations and Boussinesq wave equations, respectively. Models that are based on shallow-water equations have been applied broadly to ocean-scale tsunami modeling and local flooding as well \citep{berger2011geoclaw, george2004numerical, george2008augmented, hu2000numerical, hubbard20022d, popinet2012adaptive, qin2018comparison, wei2013dynamic}. Mathematically, the shallow-water equations do not model the dispersion in water waves directly, while the Boussinesq wave equations include an explicit dispersive term. Thus, many models based on Boussinesq wave equations have also been used \citep{kim2009depthintegrated, kim2017boussinesq, lynett2010application, madsen2003boussinesqtype, madsen1992new, shi2012highorder}. 

As computational power has grown, three-dimensional models based on RANS and LES equations have been applied for modeling of near-shore waves and floods, and, in particular, for fluid impact on coastal structures like bridges and buildings, which are relatively smaller in scales \citep{biscarini2010computational, choi2007threedimensional, larsen2017tsunamiinduced, mayer2000simulation, montagna20113d, williams2016numerical}. In addition, three-dimensional models output the pressure field directly, which can be integrated to obtain fluid forces on structures. In contrast, two-dimensional models rely on a simplified approach to convert their output to fluid forces on structures \citep{motley2016, qin2016threedimensional, qin2018comparison, sarfaraz2017sph}.

Many of these models are built into mostly open-source software packages that are broadly used by the NHE community and maintained by researchers at research institutes. Examples include GeoClaw \citep{berger2011geoclaw}, MOST \citep{titov1997implementation}, and Tsunami-HySEA \citep{macias2016comparison}.

\section{Major Research Gaps and Needs}
\label{sec:tsunami_gaps}

One challenge in tsunami modeling is to develop models of different fidelity to satisfy different needs. For instance, site-specific inundation modeling and analysis often need to be performed in the design of vertical evacuation structures \citep{ash2015structures, gonzalez2013tsunami}. In this case, a more accurate but time-consuming three-dimensional model is desired. On the other hand, compiling tsunami hazard maps---where typically thousands of runs are needed---might require using a faster but less accurate two-dimensional model.  Because ASCE 7 requires site specific analysis to predict local flow characteristics for critical structures, a more formal methodology for proper probabilistic tsunami hazard analysis over large areas is a critical research gap.

Another demand in the area is to update or even re-design the relevant software to capitalize on the rapidly growing computational power. These computational resources often require running code on clusters or newer machines with graphics processing units (GPUs); thus, there is a need to adapt these software packages to take advantage of these high-performance computing (HPC) machines.  Increased use of three-dimensional models is directly associated with increased computational power; unfortunately, three-dimensional analysis is still not computationally practical at larger scales.  This requires the development of complex coupling mechanisms between the two-dimensional and three-dimensional solvers.  Similar coupling mechanisms between the inundation models and the built environment, thus enabling both load prediction and structural response and potential debris flows, are also a necessary step in streamlining the analysis process across the various scales.  

%\end{document}
