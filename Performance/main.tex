
\begin{partbacktext}
\part{Performance Assessment}\label{part:Performance}

The built environment is a collection of various types of assets that affect the well-being and quality of life of residents in an urban area. The list of such assets includes residential and commercial buildings, bridges, networks of roads, railways, pipelines and power lines, and their supporting facilities. This part of the report focuses on evaluation of the damage and direct consequences that typically describe the immediate impact of the disaster and assign a performance measure to the various assets that make up the built environment. Part \ref{part:recovery} focuses on the simulation of recovery and consequences that consider interdependencies between infrastructure systems, the built environment, and local communities.

The performance of assets depends heavily on the description of the hazard and the simulation of asset response to one or more characteristic events that are consistent with the hazard at the asset location. Although this chapter focuses on performance assessment, some of the tools listed here have hazard and response estimation capabilities as well. Those features have been covered in Parts \ref{part:hazard} and \ref{part:response}, respectively. 

This part is organized around the types of assets and asset-networks needed to arrive at a comprehensive description of the performance of an urban region subjected to a natural-hazard event. The content is divided into four chapters to facilitate navigation, but the importance of an integrated assessment is emphasized by highlighting the synergies across assets and hazards throughout the chapters.

\end{partbacktext}
