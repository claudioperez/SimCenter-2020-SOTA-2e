
\begin{partbacktext}
\part{Performance Assessment}\label{part:Performance}

The built environment is a collection of various types of assets that affect the well-being and quality of life of residents in an urban area. The list of such assets includes residential and commercial buildings, bridges, networks of roads, railways, pipelines and power lines, and their supporting facilities. The performance of these assets is quantified by Decision Variables (DV) that describe consequences of asset damage on the life of the affected community. These DVs are meant to drive decision- and policy making. This part of the report focuses on evaluation of the damage and direct consequences that typically describe the immediate impact of the disaster and assigns a performance measure to each asset. Part \ref{part:recovery} focuses on the simulation of recovery and consequences that consider interdependencies between infrastructure systems, the built environment, and local communities. 

The performance of assets heavily depends on the description of the hazard and the simulation of asset response to one or more characteristic events that are consistent with the hazard at the asset location. Although this chapter focuses on performance assessment, some of the tools listed here have hazard and response estimation capabilities as well. Those features have been covered in the Part \ref{part:hazard} and \ref{part:response}, respectively. This part is organized around the types of assets or asset-networks needed to arrive at a description of the performance of an urban region under natural hazards. 

Seismic performance assessment of buildings has received a lot of attention from the research community and funding agencies in the past few decades \citep{atc1985atc13, fema1997guidelines, fajfar2004performancebased, kircher2006hazus}. Consequently, the most sophisticated and mature methods are available in that area \citep{atc2012p-58}. Several researchers have focused on adopting these methods for other asset types \citep{werner2006redars, chmielewski2016response} and for other types of hazards \citep{vickery2006hazus, bernardini2015performance, attary2017performancebased, barbato2013performancebased, lange2014application}. This is a challenging task because modeling damage and subsequent consequences can be fundamentally different for non-building assets and for non-seismic disasters.  
The consideration of multiple hazards and multiple asset types is the final goal for performance assessment of the built environment. It is important to separate the concepts of multiple hazard studies and multi-hazard studies. Following the naming convention suggested by \citet{bruneau2017state}, the former studies consider multiple, independent hazards in an area, while the latter studies consider the interactions and cascading effects among those hazards as well. In a multi-hazard analysis, the majority of the resulting risk is not from concurrent extreme events because the probability of two of such events happening simultaneously is very low. The characterization and modeling of more frequent hazards becomes more important and influential to the results. Multi-hazard studies already cover a wide range of hazard and asset types: earthquake mainshock and aftershock (e.g., \cite{nazari2015effect, zhang2013damage}); ground shaking, liquefaction, and landslides (e.g., \cite{elgamal2008three, kojima2014large}); earthquake and tsunami (e.g., \cite{akiyama2014reliability, carey2019multihazard}); hurricane wind, windborne debris, storm surge (e.g., \cite{lin2010windborne, park2014abv}), and rainwater (e.g., \cite{pita2012assessment}); flood and sea level rise (e.g., \cite{hinkel2014coastal}).

\end{partbacktext}
