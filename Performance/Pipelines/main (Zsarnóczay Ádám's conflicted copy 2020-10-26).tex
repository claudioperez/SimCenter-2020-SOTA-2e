%%%%%%%%%%%%%%%%%%%% author.tex %%%%%%%%%%%%%%%%%%%%%%%%%%%%%%%%%%%
%
% sample root file for your "contribution" to a contributed volume
%
% Use this file as a template for your own input.
%
%%%%%%%%%%%%%%%% Springer %%%%%%%%%%%%%%%%%%%%%%%%%%%%%%%%%%%%%%%%%


%% RECOMMENDED %%%%%%%%%%%%%%%%%%%%%%%%%%%%%%%%%%%%%%%%%%%%%%%%%%%
%\documentclass[graybox]{svmult}
%
%% choose options for [] as required from the list
%% in the Reference Guide
%
%\usepackage{mathptmx}       % selects Times Roman as basic font
%\usepackage{helvet}         % selects Helvetica as sans-serif font
%\usepackage{courier}        % selects Courier as typewriter font
%\usepackage{type1cm}        % activate if the above 3 fonts are
                             % not available on your system
%
%\usepackage{makeidx}         % allows index generation
%\usepackage{graphicx}        % standard LaTeX graphics tool
%                             % when including figure files
%\usepackage{multicol}        % used for the two-column index
%\usepackage[bottom]{footmisc}% places footnotes at page bottom
%
%% see the list of further useful packages
%% in the Reference Guide
%
%\makeindex             % used for the subject index
%                       % please use the style svind.ist with
%                       % your makeindex program
%
%%%%%%%%%%%%%%%%%%%%%%%%%%%%%%%%%%%%%%%%%%%%%%%%%%%%%%%%%%%%%%%%%%%%%%%%%%%%%%%%%%%%%%%%%%
%
%\begin{document}

\title{Water, Sewer, and Gas Pipelines}
% Use \titlerunning{Short Title} for an abbreviated version of
% your contribution title if the original one is too long
\author{
    \textbf{Iris Tien}
    \and Jack W. Baker
    \and Adam Zsarnóczay}
\tocauthor{}
\authorrunning{Tien et al.}
% Use \authorrunning{Short Title} for an abbreviated version of
% your contribution title if the original one is too long
%\institute{Name of First Author \at Name, Address of Institute, %\email{name@email.address}
%\and Name of Second Author \at Name, Address of Institute %\email{name@email.address}}
%
% Use the package "url.sty" to avoid
% problems with special characters
% used in your e-mail or web address
%
\maketitle

There are many methods available for modeling and simulation of lifeline networks. These include empirical, agent-based, system dynamics-based, economics theory-based, and network-based approaches. The reader is referred to resources including Ouyang (2014) and Johansen et al. (2016) for a discussion of the general methods available and the varying measures existing for assessing community outcomes based on lifeline performance. The techniques described in the works referenced are general and can be applied to any lifeline network; they are not described in more detail here. The focus in this report is on modeling and simulation software for specific lifeline types with parameters particular to the lifeline and resource flow type. The emphasis is on open and publicly available software tools.

The objective of water, sewer, and gas pipeline network simulation is to assess the ability to provide critical water, wastewater, and natural gas services for populations under varying scenarios. As relevant for natural hazards engineering, the purpose is to assess the states of these lifeline systems under hazard events and inform decision making on approaches to improve the expected performance of these systems when subjected to these hazards. The underlying physics of water, sewer, and gas pipeline simulation lie mainly with the simulation of individual components in the system, e.g., pipes, junctions, and valves, and subsequently system-level analysis based on the component-level information through the use of network graphs or by more detailed hydraulic flow and pressure analysis through the network.
 
\section{Input and Output Data}
\label{sec:perf_pipeline_io}

The information needed to describe the natural hazard at a regional scale is similar to the hazard characterization explained for transportation networks in the previous section. The hazard models provide a regional distribution of IMs that are used as proxies to express the severity of the hazard in the area. The type of IM depends on the hazard type and the network component under investigation. While the effect of a seismic event on structures is typically described using PGA or spectral acceleration, the analysis of pipelines requires information about the PGV and the permanent ground deformation (Romero et al., 2010).

Lifeline models require information about the geographical location and classification of the lifeline components. Such information is often hard to obtain, especially at a sufficiently high resolution for detailed regional assessment. This lack of data stems partly from privacy and national security concerns and partly from the fact that the databases are privately owned. Water, sewer, and gas pipeline simulation relies on information about component locations, size (e.g., pipe diameter), pressure, and connectivity. 

Output data will typically be water, wastewater, or gas pressures, flows, or volumes at specific locations. Of particular interest is the ability to provide these services at final distribution points in the network.

\section{Modeling Approaches}
\label{sec:perf_pipeline_methods}

The modeling approach in the simulation is often determined and constrained by the amount and resolution of available information. Regardless of the modeling fidelity, the approach is almost always based on the following two steps: 

\begin{itemize}
    \item First, given an IM at the site, component damage is estimated by assigning a DS to each component. In HAZUS, this is done using component-specific fragility curves to estimate the damage to a given component under the hazard. For facilities and building-like structures, the damage evaluation is similar to the HAZUS method described in Section 3.1. For pipeline networks (i.e., water, sewer, and gas), two types of damages are considered: leaks and breaks. The sophistication of damage evaluation heavily depends on the level of analysis and the available IMs. 
    \item The second step, given estimated damage to lifeline components, is to assess network-level consequences. In HAZUS, the direct consequences are evaluated using empirical relationships based on past experience. Direct consequences are typically limited to restoration time and replacement cost in HAZUS.
\end{itemize}

The HAZUS earthquake model is the most sophisticated among the HAZUS models for lifeline modeling and simulation. For natural hazards engineering, HAZUS hurricane and tsunami models are limited to buildings and do not cover lifelines. The HAZUS flood model provides damage and loss estimates only for a subset of potable water, wastewater, and natural gas facilities. Damage is a function of flooding as measured by water level in feet. The information below is based on the earthquake hazard modeling.

The default inventory for potable water networks in HAZUS contains estimates of pipelines aggregated at the census tract level (based on U.S. Census TIGER street file datasets). The HAZUS methodology suggests three levels of analysis for potable water networks:

\begin{itemize}
    \item Level 1 is based on the default HAZUS inventory (i.e., census tract estimates).
    \item Level 2 requires additional information on transmission aqueducts, distribution pipelines, reservoirs, water treatment plants, wells, pumping stations, and storage tanks. 
    \item Level 3 requires additional information on junctions, hydrants, and valves, and further data about connectivity and serviceability (i.e., demand pressures, and flow demands at different distribution nodes). Such information is typically available in KYPIPE, EPANET, or CYBERNET format.
\end{itemize}

\noindent Analysis of other lifelines requires similar types of information:

\begin{itemize}
    \item Wastewater networks are described by the geographical layout and characteristics of the transmission network and its treatment components.
    \item Natural gas networks are described by the geographical layout and characteristics of buried or elevated pipelines and compressor stations.
\end{itemize}

The diameter of pipes is not considered as a parameter in the damage functions. However, it may be a good proxy for capacity of the given network element and used in network performance assessment in more sophisticated analyses. The rigidity of the pipes in a network is an important input parameter that heavily influences the damage to the pipelines by earthquakes. 

\noindent In HAZUS, three levels of modeling fidelity are described that correspond to the previously introduced analysis level: 

\begin{itemize}
    \item Level 1: Results are limited to number of leaks and breaks per census tract resulting in a simplified evaluation of network performance (i.e., total number of households without water).
    \item Level 2: The network is modeled as a graph. This approach allows for estimates of component functionality, component damage ratio, and flow reduction to each area served by the network. Overall network performance can be estimated as a function of the average repair rate (repairs/km) of the pipes in the network. Such evaluations have been performed for San Francisco (Markov, Grigoriu, O’Rourke, 1994), Oakland (G\&E, 1994), and Tokyo (Isoyama and Katayama, 1982).
    \item Level 3: The suggested model is based on the work of Khater and Waisman (1999). It provides more accurate estimates of the hydraulic flow in the network. This translates into improved accuracy and reliability of results when compared to Level 2 analyses.
\end{itemize}

\section{Software Systems}
\label{sec:perf_pipeline_tools}

\noindent\textbf{HAZUS 4.2} \\This methodology classifies potable water, wastewater, oil, natural gas, electric power, and communication systems as lifelines. It provides a similar methodology for the modeling and simulation of these systems. Electrical networks are the only lifelines that have influence on other lifelines in the HAZUS methodology (i.e., damage to the electrical network and the consequent loss of power results in loss of function and potential damage in other lifelines). Electrical system modeling is described in more detail in Section 3.4.

The dependency between network component repair times is not considered by HAZUS. The dependencies of one lifeline damage or the consequences of such damage on other lifelines is not taken into consideration in the HAZUS methodology, with the exception of electrical networks. Assessment of network performance is out of the scope of the HAZUS methodology for wastewater systems. In general, HAZUS 4.2 allows estimation of lifeline damage and consequent reduction in network performance. Further details of the software and its limitations are explained in Section 1.1.
\newline

\noindent\textbf{EPANET} \\EPANET is a software package developed by the U.S. Environmental Protection Agency (EPA) for water pipeline distribution modeling and simulation. It has been widely adopted by municipalities and water utilities as a standard format to evaluate their systems. Its two main uses are for hydraulic modeling, including for maintaining flows and pressures in a system, and for contaminant transport simulation. EPANET files can be used as input files for water distribution pipeline information. EPANET modeling and simulation does not include the ability to assess the impacts of natural hazards on lifeline components or network performance.
\newline

\noindent\textbf{GIRAFFE} \\The Graphical Iterative Response Analysis for Flow Following Earthquakes (GIRAFFE) was developed at Cornell University to provide a performance assessment tool for pipeline networks (Wang and O’Rourke, 2008). It uses the EPANET engine to define the water network and perform hydraulic network analysis. Performance estimates are presented in a user interface using a GIS framework.
\newline

\noindent\textbf{WNTR} \\WNTR (Water Network Tool for Resilience) is an open-source library of functions developed in the Python programming language by Sandia Laboratories. It uses input data in the EPANET format to define a network and perform an analysis that is similar to the Level 2 and 3 analyses in the HAZUS methodology. WNTR adds the hazard element to water pipeline simulation. Using WNTR requires basic Python programming skills, but in return it provides a platform-independent solution that can easily work in a HPC environment. In addition to the damages and estimated restoration times, it provides estimates of the hydraulic performance of the damaged network. The library is hazard-agnostic; as long as the IMs and the corresponding fragility curves are supplied, it performs the damage calculations and evaluates the estimated consequences of the damage.
\newline

\noindent\textbf{Gas pipeline simulation software} \\Compared to water pipeline simulation software, natural gas pipeline modeling and simulation software is mostly commercial and proprietary, e.g., Synergi Gas from DNV-GL that conducts hydraulic modeling and analysis and NextGen from Gregg Engineering to create hydraulic simulation models to run simulations and calculate pressures, flow rates, and other operational parameters. Given the emphasis on open-source software in this report, these are not discussed in more detail.

\section{Research Gaps and Needs}
\label{sec:perf_pipeline_gaps}

In lifeline modeling and simulation, there is the need for improved consideration of interdependencies between lifelines. This will enable researchers to better understand the impacts of natural hazards on communities. Advancing component-level fragility curves and improved simulation of structural response for critical facilities will lead to a better estimate of expected damage, both in terms of accuracy and resolution in a simulation. Finally, improved simulation of the recovery process will enable researchers to better understand performance and recovery of lifeline services during and after disasters.