%%%%%%%%%%%%%%%%%%%% author.tex %%%%%%%%%%%%%%%%%%%%%%%%%%%%%%%%%%%
%
% sample root file for your "contribution" to a contributed volume
%
% Use this file as a template for your own input.
%
%%%%%%%%%%%%%%%% Springer %%%%%%%%%%%%%%%%%%%%%%%%%%%%%%%%%%%%%%%%%


%% RECOMMENDED %%%%%%%%%%%%%%%%%%%%%%%%%%%%%%%%%%%%%%%%%%%%%%%%%%%
%\documentclass[graybox]{svmult}
%
%% choose options for [] as required from the list
%% in the Reference Guide
%
%\usepackage{mathptmx}       % selects Times Roman as basic font
%\usepackage{helvet}         % selects Helvetica as sans-serif font
%\usepackage{courier}        % selects Courier as typewriter font
%\usepackage{type1cm}        % activate if the above 3 fonts are
                             % not available on your system
%
%\usepackage{makeidx}         % allows index generation
%\usepackage{graphicx}        % standard LaTeX graphics tool
%                             % when including figure files
%\usepackage{multicol}        % used for the two-column index
%\usepackage[bottom]{footmisc}% places footnotes at page bottom
%
%% see the list of further useful packages
%% in the Reference Guide
%
%\makeindex             % used for the subject index
%                       % please use the style svind.ist with
%                       % your makeindex program
%
%%%%%%%%%%%%%%%%%%%%%%%%%%%%%%%%%%%%%%%%%%%%%%%%%%%%%%%%%%%%%%%%%%%%%%%%%%%%%%%%%%%%%%%%%%
%
%\begin{document}

\title{Uncertainty Quantification}
% Use \titlerunning{Short Title} for an abbreviated version of
% your contribution title if the original one is too long
\author{
    \textbf{Alexandros Taflanidis}
    \and Joel P. Conte
    \and George Deodatis
    \and Sanjay Govindjee}
\tocauthor{}
\authorrunning{Taflanidis et al.}
% Use \authorrunning{Short Title} for an abbreviated version of
% your contribution title if the original one is too long
%\institute{Name of First Author \at Name, Address of Institute, %\email{name@email.address}
%\and Name of Second Author \at Name, Address of Institute %\email{name@email.address}}
%
% Use the package "url.sty" to avoid
% problems with special characters
% used in your e-mail or web address
%
\maketitle

Uncertainty quantification (UQ) represents one of the fastest evolving scientific fields, with advances in computer science and statistical computing promoting constant developments in the way uncertainty is incorporated in the predictive analysis of engineering systems (Smith, 2013). In particular, over the last decade(s) the popularity of HPC and of machine-learning tools have dramatically impacted the way computational simulation is utilized within the UQ field, lifting many barriers that were traditionally associated with simulation-based UQ techniques, and allowing the detailed characterization and propagation of uncertainties even for problem with highly complex (computationally intensive) numerical models. . It is the current consensus within the UQ community that these advances will (or have already) remove(d) the need for simplified approaches with respect to both the uncertainty characterization (assumptions/models used to describe uncertainty and system performance) or uncertainty propagation (estimation of statistics of interest). 

When discussing computational advances and state-of-the art tools in UQ, greater emphasis is typically placed on algorithmic approaches rather than the corresponding software facilitating the implementation of these approaches. The reason for this is that development of scientific tools for UQ has focused traditionally on a specific UQ sub-field [for example, surrogate modeling to support UQ analysis (Lophaven et al. 2002; Gorissen et al. 2010)], with a large number of researchers [for example, (Bect et al. 2017; Clement et al. 2018)] offering open-source algorithms to even address a specific class of problems within each of these sub-fields. Many of these algorithms have been developed in MATLAB, though in recent years significant emphasis has been placed on open-source libraries developed using Python (or less frequently C++ or R) and typically distributed though GitHub.    

Since UQ is a very broad field, here discussions focus on applications within the natural hazards engineering field, with some references to relevant general UQ advances also offered. Emphasis is on computational aspects, the most pertinent UQ feature for a state-of-the-art review of UQ simulation methods. Additionally, discussions focus on algorithmic developments, with some references also on relevant software. With respect to description of uncertainty emphasis is placed on probabilistic characterization; even though alternative approaches exist (Beer et al. 2013), such as use of fuzzy sets and interval analysis, the current standard of practice in natural hazards engineering is to rely on probabilistic UQ analysis. This can be attributed to the tradition in civil engineering codes to describe performance with respect to statistical measures (probability of exceeding performance limit states), or to the fact that hazard exposure, the most significant source of variability when discussing risk in the natural hazards engineering context, is almost always described using probabilistic measures (McGuire, 2004; Resio et al., 2007). 

Should be stressed that all references provided here are merely indicative ones, though few of them can be regarded as seminal work, since the UQ field, even with respect to natural hazards engineering applications, is very broad and constantly expanding.    

\section{Uncertainty Characterization}
\label{sec:uq_characterization}

In natural hazards engineering, characterization of the uncertainties impacting predictions is integrally related to risk quantification. Performance based engineering (PBE) (Whittaker et al. 2003; Goulet et al. 2007; Riggs et al. 2008; Ciampoli et al. 2011; Barbato et al. 2013; Fischer et al. 2019) represents undoubtedly the foundational development for this task. Performance-based engineering decouples the risk quantification to its different components, mainly hazard analysis (exposure), structural analysis (vulnerability), and damage and loss analysis (consequences), with uncertainties included (and impacting predictions) in all these components. Variability of the hazard itself, in terms of both occurrence and intensity, is widely acknowledged to correspond to the most significant source of uncertainty in this setting. Frequently hazard variability is represented through a resultant IM (Baker and Cornell, 2005; Kohrangi et al., 2016), though comprehensive approaches that focus on connecting the excitation to parameters of the geophysical process creating it also exist, for example in earthquake engineering description of time-histories through use of stochastic ground motion models dependent on seismological parameters (Bijelić et al., 2018; Vlachos et al., 2018) or in coastal risk estimation use of surge modeling numerical tools dependent on atmospheric storm characteristics (Resio et al., 2007). Beyond the hazard variability, uncertainties related to parameters of the structural model or generalized system model (for applications not examining directly structural risk) and to the characteristics for describing performance are also recognized as important for inclusion in risk estimation (Porter et al., 2002). The term “system” will be used herein to describe the application of interest; this may pertain, for example, to a building model, to an infrastructure network, or to a soil–structure interaction system configuration. 

Uncertainty within this natural hazards engineering risk characterization setting is ultimately described through a discrete number of parameters (treated as random variables) pertaining to either the hazard or the system/performance model,including parameters to describe interdependencies and deterioration characteristics (Jia and Gardoni 2018; Akiyama et al. 2020). Even when the uncertainty description for the underlying problem actually entails a stochastic sequence/process or a random field, a discretized approximation of these functions is commonly utilized, as necessitated by the numerical tools used to compute the system response (Gidaris et al., 2014). This translates into use of a parameterized realization for the excitation or model characteristics, an approach that seamlessly fits within the overall PBE framework. Exceptions exist primarily for stochastic dynamics problems, for which propagation of the stochastic excitation uncertainty can be performed using random vibration theory, such as exact or approximate solution of stochastic differential equations or estimation of stationary statistics in the frequency domain (Li and Chen, 2009). Though such approaches offer substantial benefits, their implementation is primarily constrained to linear systems or nonlinear systems with moderate degree of complexity (dos Santos et al., 2016; Wang and Der Kiureghian, 2016), such as systems with very small number of degrees-of-freedom or nonlinearities having simple, analytical form. As such, their utility within natural hazards engineering is limited to specialized applications. Even in such cases, the remaining uncertainties, beyond the stochastic excitation itself, must be described using a parametric description. The overall parameterized uncertainty description promoted within PBE is therefore well aligned with such approaches, as their adoption simply requires substitution of the deterministic simulation system model with a stochastic simulation system model, the latter representing the solution of the stochastic dynamics problem.

When using Monte Carlo simulation techniques to propagate uncertainty (see discussion in the next paragraph), a critical part of the methodology is the numerical generation of sample functions of the stochastic processes, fields, and waves involved in the problem, modeling the uncertainties in the excitations (e.g., wind velocities, seismic ground motion, and ocean waves) and in the structural system (e.g., material, mechanical and geometric properties). These processes, fields, and waves can be stationary or non-stationary, homogeneous, or non-homogeneous, scalar or vector, 1D or multi-dimensional, Gaussian or non-Gaussian, or any combination of the above. It is crucial for a simulation algorithm to be computationally efficient as a very large number of sample functions might be needed. A wide range of methodologies is currently available to parametrically describe uncertainty and perform these simulations including the spectral representation method (Li and Kareem 1991; Shinozuka and Deodatis 1991; Shields et al. 2011; Benowitz and Deodatis 2015), Karhunen-Loeve expansion and polynomial chaos decomposition (Ghanem and Spanos 1991), auto-regressive moving-average models (Spanos 1983; Deodatis and Shinozuka 1988), local average subdivision method (Fenton and Vanmarcke 1990), wavelets (Zeldin and Spanos 1996), Hilbert transform techniques (Wang et al. 2014), and turning bands methods (Mantoglou and Wilson 1982).

The setting outlined in the previous two paragraphs leads, ultimately, to risk characterized as a multidimensional integral over the parametric uncertainty description (input), with uncertainty propagation (output) translating to estimation of the relevant statistics (estimation of integrals representing moments or failure probabilities with respect to different limit states). Though proper definition of the statistics of interest, different attitudes towards risk can be furthermore described, including risk aversion which is especially relevant for natural hazards engineering applications (Cha and Ellingwood 2012). The aforementioned integral is frequently expressed with respect to the conditional distributions of the different resultant risk components (Goulet et al. 2007; Barbato et al. 2013), for example {hazard / response given hazard / consequences given response}. This represents merely a simplification for risk quantification purposes as allows for the decoupling of the different components. Even when this simplification is invoked, risk fundamentally originates from the uncertainty in the model parameters of the problem formulation (including hazard/system/performance description), quantified by assigning a probability distribution to them, representing the UQ input. It should be noted that in many natural hazards engineering applications some aspects of the performance (and of the associated risk) are described by utilizing resultant statistical models instead of explicitly addressing the underlying sources of uncertainty; for example, in PBE fragility functions are frequently leveraged to describe the combined effect of the uncertainties influencing the parameters of capacity and demand models, while in loss estimation resultant distributions are leveraged to encompass the multiple sources of uncertainty influencing consequence quantification. In such cases the integral quantifying risk is expressed with respect to the remaining parametric uncertainty sources.      . 

\section{Uncertainty Propagation}
\label{sec:uq_propagation}

For uncertainty propagation, the traditional approach in natural hazards engineering has been the use of point estimation methods, either methods that focus on the most probable values of the model parameters like the first-order second moment (FOSM) method (Baker and Cornell, 2008) and its variants (Vamvatsikos, 2013), or methods that focus on the peaks of the integrand of the probabilistic integral (design points) like the first- and second-order reliability methods (FORM/SORM) (Koduru and Haukaas, 2010). As point estimation methods are inherently approximate, with no available means to control their accuracy, advances in computer science and statistics, including the development of innovative MCS techniques, such as Subset Simulation (Au and Beck 2003b), Efficient Global Reliability Analysis (EGRA) (Bichon et al. 2013), and Adaptive Kriging with Monte Carlo Simulation (AK-MCS) (Echard et al. 2011), have encouraged researchers the past decade to rely more heavily on Monte Carlo simulation (MCS) tools for uncertainty propagation in NHE (Smith and Caracoglia 2011; Taflanidis and Jia 2011; Vamvatsikos 2014; Esposito et al. 2015; Deb et al. 2019a). 

Although point estimation methods do still maintain utility and popularity, NHE trends follow the broader UQ community trends in promoting computer and Monte Carlo simulation approaches, as these techniques facilitate high accuracy uncertainty propagation (unbiased estimation) with limited fundamental constraints on the complexity of the probability and numerical models used. Of course, computational complexity is still a concern for MCS, especially for applications with high-dimensional uncertainties and challenging quantities of interest (such as rare event simulation), creating in many instances practical constraints for the efficient implementation. The current state of the art in NHE for addressing these constraints is to leverage both advanced MCS techniques (Au and Beck 2003b; Li et al. 2017; Bansal and Cheung 2018) but, more importantly, machine learning and advanced computational statistics tools (Echard et al. 2011; Abbiati et al. 2017; Ding and Kareem 2018; Su et al. 2018; Wang et al. 2018). Relevant recent advances for MCS focus on variance reduction techniques, e.g., Latin hypercube sampling (Vamvatsikos 2014), stratified sampling (Jayaram and Baker 2010), importance sampling (Papaioannou et al. 2018), Markov chain Monte Carlo methods (Au and Beck 2003b) and sequential approaches for rare event simulation (Jia et al. 2017), with substantial emphasis on problems with high-dimensional uncertainties (Au and Beck 2003b; Wang and Song 2016), whereas for machine learning focus is primarily on use of a variety of surrogate modeling (metamodeling) techniques (Stern et al. 2017; Zhang and Taflanidis 2018b; Bernier and Padgett 2019; Gentile and Galasso 2020; Le and Caracoglia 2020; Zhang et al. 2020). Many machine learning implementations in natural hazards engineering fall under the category of direct adoption of techniques developed by the broader UQ community, though number of studies do address challenges unique to the integration of surrogate modeling in natural hazards engineering problems, for example the need to address high-dimensionality of input when stochastic description is utilized for non-stationary excitations (Gidaris et al. 2015).  

Note: the natural hazards engineering modeling community has been continuously increasing the complexity of the models they adopt. Such high fidelity numerical models, able to capture the behavior of structural, geotechnical and soil-foundation-structural systems all the way to collapse or the brink of collapse, are inherently nonlinear hysteretic (path-dependent) and frequently degrading/softening and therefore present (significant) challenges in term of robustness of convergence of the iterative schemes used to integrate their equations of motion. The significance of these challenges will further increase in the MCS based UQ context and requires significant research efforts to be overcome. One of the implicit outcomes of this complexity increase, is a further increase of the input/output dimensionality in natural hazards engineering UQ problems. This leads to applications with high-dimensional uncertainties that, traditionally, pose a challenges for UQ algorithms (Au and Beck 2003a; Schuëller et al. 2004). Different approaches have been explored to address this challenge within natural hazards engineering applicaitons, ranging from specialized MCS algorithms (Au and Beck 2001; Wang and Song 2016), to the formal integration of dimensional reduction techniques (Jia and Taflanidis 2013), to the use of global sensitivity analysis indices to the selection of subsets of inputs to emphasize in different MCS schemes (Jia et al. 2014). It is evident, though, that stronger emphasis will be required on this topic in the future.     

Discussing more broadly advances in the UQ field, emphasis is currently strongly placed on machine learning techniques for accelerating UQ computations (Murphy 2012; Ghanem et al. 2017; Tripathy and Bilionis 2018). The relevant developments are frequently integrated with advanced MCS techniques, for example for topics like rare event simulation (Li et al. 2011; Balesdent et al. 2013; Bourinet 2016). With respect directly to machine learning, some emphasis is given on the approaches for tuning and validation (Mehmani et al. 2018), though the primary focus is on the proper design of the computer simulation experiments (DoE) (Kleijnen 2008; Picheny et al. 2010; Kyprioti et al. 2020) that are used to inform the development of the relevant computational statistics tools. Adaptive DoE is widely acknowledged to offer substantial advantages in balancing computational efficiency and accuracy for UQ analysis when machine learning techniques are used, and significant research efforts are currently focused on advancing DoE techniques, remaining an open challenge for the community. Should be noted that characteristics of the adaptive DoE depend on the utility of the surrogate model (Liu et al. 2018), whether is it intended to serve as a global replacement of the original numerical model (i.e., develop the surrogate model first and then leverage it to perform different UQ tasks) or be used for a very specific UQ tasks (for example estimate reliability index for a specific limit state). 

The concept of model fidelity remains unexplored within the natural hazards engineering community, but it plays a central role in modern UQ techniques, with a range of algorithms developed to properly integrate hierarchical fidelity models to promote efficient and accurate uncertainty propagation (Geraci et al., 2017; Peherstorfer et al., 2018). Combination of machine-learning (primarily surrogate modeling) techniques with different fidelity models is also a topic that has been receiving increasing attention for facilitating the use of expensive numerical models in UQ (de Baar et al., 2015; Zhou et al., 2016). In the natural hazards engineering setting, discussions on explicitly exploiting model fidelity for risk estimation are very limited; therefore, the community still heavily emphasizes use of high-fidelity models without, yet, examining how different levels of simulation fidelity and the use of reduced order models can be properly combined to promote efficient and accurate risk estimation. Multi-fidelity Monte Carlo and hierarchical surrogate modeling techniques constitute, undoubtedly, important opportunity areas for advancing UQ analysis in natural hazards engineering.

Another important aspect of uncertainty propagation is the concept of sensitivity analysis. In natural hazards engineering this has been primarily implemented as local sensitivity analysis, i.e., estimation of gradient information (Haukaas and Der Kiureghian 2007; Gu et al. 2009),  since this fits well with the point estimation methods used frequently for calculation of statistics (helps the identification of design points). The more relevant, though, within a UQ setting is global sensitivity analysis (Sobol' 1990; Saltelli 2002; Rahman 2016), as it allows identification of the relative importance of the different sources of uncertainty, offering insights with respect to both accelerating UQ computations (helping, in particular, with high-dimensional applications as discussed earlier), as well as to understanding of the critical factors impacting the overall risk. Though global sensitivity analysis can be particularly useful for hazard applications (Vetter and Taflanidis 2012), it is currently receiving limited practical interest within natural hazards engineering, though implementations do exist even for all purpose codes, for example (Bourinet et al. 2009). More formal integration of global sensitivity analysis tools within the natural hazards engineering community represents, therefore, another topical area that advancements can/should be made. Of course the computational cost for global sensitivity analysis, for example calculation of first and higher order sensitivity indexes, is much higher than the cost of simple uncertainty propagation; relevant techniques range from use of quasi-Monte Carlo (Saltelli 2002) to surrogate modeling (Sudret 2008) to sample-based methods relying on approximation of conditional distributions (Li and Mahadevan 2016; Hu and Mahadevan 2019).  

\section{Model Calibration and Bayesian Inference}
\label{sec:uq_calibration}

Model updating/calibration plays an important role in natural hazards engineering, with data coming from both component (or system) –level experiments or system–level observations during (or post) actual excitation conditions. Within UQ setting, the current standard to perform this updating is Bayesian inference (Beck 2010; Kontoroupi and Smyth 2017). Using observation data, Bayesian inference can be leveraged to provide different type of outputs/results (Beck and Taflanidis 2013) through the following three tasks: identify the most probable model parameters or even update the entire probability density function for these parameters (obtain posterior distributions); perform posterior predictive analysis and update risk using the new information; when different numerical models are examined, identify the probability of each of them (as inferred by the data) to either select the most appropriate or calculate the weights when all of them will be used in a model averaging setting (model class selection). The typical implementation refers to model parameter updating, what is traditionally viewed as model calibration, with model class selection less frequently used, especially within natural hazards engineering community applications. Still, Bayesian model class selection offers a comprehensive tool for evaluating appropriateness of different models (Muto and Beck 2008), and especially for natural hazards engineering applications can be integrated with health monitoring tools (Oh and Beck 2018). Ideally, model parameter updating should be performed accounting for pertinent sources of real-world uncertainties (e.g.: noisy input-output measurements, uncertainty in model parameters, model form uncertainty, environmental variability). Formulating the likelihood equation to account for such uncertainties is a crucial part of the Bayesian model updating/calibration process. The so-called Kennedy O’Hagan framework has emerged as a robust approach to account for all these uncertainties, especially the model form error (Kennedy and O'Hagan 2001). 

From computational perspective, Bayesian updating can have significant computational burden, especially when complex FEM models are utilized. Such FEM models contain numerous unknown parameters which drastically increases the computational cost of the Bayesian updating process. This necessitates the use of identifiability and sensitivity analyses prior to model updating to select the most significant/influent parameters to use in the updating process. This step tremendously improves the calibration process runtime of complex FE models with numerous parameters and results in better parameter estimation results (Ramancha et al. 2021b). In addition, non-identifiable parameters result in non-unique parameter estimates (Ramancha et al. 2020). Fisher information based local identifiability analysis and variance-based global sensitivity analysis are commonly used methods for parameter screening to identify the most significant/influent and identifiable parameters (Ramancha et al. 2021b). Beyond this critical dimensionality reduction, a variety of algorithmic approaches are commonly used to address the computational complexity in Bayesian updating applications. Common approaches include the use of advanced MCS techniques to reduce the total number of simulations needed (Quiroz et al. 2018), the integration of metamodeling to approximate the complex system model (Angelikopoulos et al. 2015; Giovanis et al. 2017; Wang and Shafieezadeh 2019; Zhang and Taflanidis 2019) or the use of direct differentiation tools to accelerate computations (Astroza et al. 2017). The second candidate implementation, the use of surrogate modeling techniques, has undoubtedly significant potential in accommodating the use of complex models within Bayesian calibration schemes.     

Bayesian updating may rely on point estimates, equivalent to identifying and using only the most probable (based on the observation data) model parameter values - expressed as a nonlinear optimization problem - , or leverage the entire posterior distribution - expressed as problem of sampling from this distribution. For the latter, Markov Chain Monte Carlo (MCMC) techniques need to be used for any of the three tasks entailed in Bayesian inference (Catanach and Beck 2018). In particular, transitional MCMC (TMCMC) is a versatile method for sampling the posterior distribution (Ching and Chen 2007; Betz et al. 2016) and due to its computationally parallel nature, is ideal for Bayesian inference of computationally expensive FE models (Ramancha et al. 2021a; Ramancha et al. 2021b). For problems involving inference for dynamical models (which majority of applications in natural hazards engineering fall under), updating can be done in batch mode, using all observation data at once, or recursive mode, sequentially updating model characteristics during the time-history for the observations (Astroza et al. 2017; Ramancha et al. 2021b). The batch approach is a direct implementation of the broader Bayesian inference framework. The recursive implementation typically leads to filtering approaches, including Kalman filters (KF) and its variants (Extended KF or Unscented KF) that relies on linear or Gaussian assumptions (Astroza et al. 2017; Kontoroupi and Smyth 2017; Erazo and Nagarajaiah 2018), and particle filters (PF) that reproduces the work of KF in non-linear and/or non-Gaussian environment by a sequential MCS approach (Chatzi and Smyth 2009; Wei et al. 2013; Olivier and Smyth 2017). Recursive approach is used primarily for real-time or online applications, focusing mainly on the most probable parameter values. 

\section{Design under Uncertainty}
\label{sec:uq_design}

In natural hazards engineering, design under uncertainty has been traditionally expressed as a reliability-based design optimization (RBDO) (Spence and Gioffrè 2012; Chun et al. 2019) or as a robust design optimization (RDO) (Greco et al. 2015) problem. Some recent approaches deviate from this pattern and follow directly PBE advances, formulating the design problem with respect to life-cycle cost and performance objectives (Shin and Singh 2014), even adopting multiple probabilistic criteria to represent different risk-attitudes (Haukaas and Mahsuli 2012; Gidaris et al. 2017; Li and Conte 2018; Deb et al. 2019b). Practical applications focus on design of supplemental dissipative devices (Shin and Singh 2014; Gidaris et al. 2017; Altieri et al. 2018) and member-sizing (Huang et al. 2015; Suksuwan and Spence 2018), or even on topology-based optimization of structural systems (Bobby et al. 2017; Zhu et al. 2017).

With respect to the solution of the corresponding optimization problem, the natural hazards engineering community follows the broader UQ trends. Design under uncertainty optimization problems undoubtedly present significant computational challenges since they combine two tasks, each with considerable computational burden: uncertainty propagation and optimization. Discussed next is how uncertainty propagation is handled within this coupled problem.

Common approaches, especially within context of RBDO and RDO, typically rely on approximate point-estimation methods like FORM/SORM (Papadimitriou et al. 2018) and some sort of decoupling of the optimization/uncertainty-propagation loops to accelerate convergence (Beyer and Sendhoff 2007). Advances in the use of simulation techniques within UQ have created new opportunities the past decade to incorporate MCS techniques in solving design under uncertainty problems (Spall 2003; Flint et al. 2016), lifting some of the traditionally associated computational barriers. Greater emphasis is continuously placed on solving design under uncertainty problems using advanced Monte Carlo techniques (Medina and Taflanidis 2014), frequently coupled with an intelligent integration of surrogate modeling tools (Dubourg et al. 2011; Bichon et al. 2013; Zhang and Taflanidis 2018a). It is expected that this trend will continue, since computer science and machine learning advances have dramatically altered the computational complexity for leveraging MCS for design optimization under uncertainty, offering an attractive alternative to traditional approaches that relied on the approximate (but highly efficient) point estimation methods.   

\section{Relevant Software}
\label{sec:uq_tools}

Beyond specific UQ algorithms developed by individual researchers and shared in repositories like GitHub or MATLAB’s File Exchange, two other important UQ software categories exist:

\begin{itemize}
    \item Libraries integrated with existing modeling tools appropriate for natural hazards engineering analysis, like the general purpose FEM reliability tools offered through FERUM (Bourinet et al., 2009). These libraries frequently address a specific type of UQ analysis, e.g., direct MCS or reliability estimation.
    \item Software that looks at UQ analysis with a broad brush could be appropriate for use in natural hazards engineering applications (but as of yet has not necessarily been developed specifically for that purpose). Such software typically covers the entire range of UQ analysis, with continuous integration of the relevant state-of-the art advances. They are composed of scientific modules that perform different UQ tasks, connected through the main software engine and, in addition, are commonly equipped with an appropriate GUI. 
\end{itemize}
	
The last category of UQ software packages is of greater interest, especially since it covers the entire domain of a rapidly expanding field and facilitates the integration of the relevant developments, which typically leverage different classes of tools (e.g., rare-event simulation using surrogate models with adaptive refinement. UQ software programs typically address the following tasks:

\begin{itemize}
    \item Probabilistic modeling. This pertains to standard uncertainty characterization, extending from simple parametric description to stochastic characterization (including dimensionality reduction), and represents the input to the UQ software. 
    \item Monte Carlo and reliability analysis, extending from direct MCS with Latin hypercube sampling, to use of point estimation methods (FORM/SORM), to variance reduction, to rare event simulation. UQ outputs considered correspond typically to statistical moments, probabilities of exceedance for different limit states, or fitted distributions. The numerous software programs adopt different tools for the aforementioned tasks and most lack a complete adaptive implementation; some degree of competency on behalf of the end-user for selecting appropriate algorithms and parameters is assumed. Many types of software have started recently to integrate multi-fidelity MCS approaches.
    \item Surrogate modeling. Common classes of metamodels used include Gaussian processes, polynomial chaos, support vector machines, and radial basis functions. The developed surrogate models can be then leveraged within the software to accelerate computations for other UQ tasks. Adaptive DoE options are typically available and used frequently; standard DoE approaches are not, however, necessarily tailored to the specific UQ task the end-user is interested in applying. Most software programs sacrifice robustness (an approach that is reliable and works independent of the end-user competency) for efficiency (the ability to develop high-accuracy metamodels with the least number of simulation experiments).
    \item Global sensitivity analysis. This is typically performed through calculation of Sobol indices (UQ output) using some approximate (quasi-Monte Carlo) technique or surrogate modeling (polynomial chaos expansion). 
    \item Data analysis and model calibration, with some emphasis on Bayesian inference techniques. Although Bayesian updating is very common, model class selection is not. Addressing modeling complexity remains a bigger challenge for Bayesian inference applications since integrating metamodeling techniques is not trivial. The challenge here is to establish a fully automated integration that can address different degrees of competency for the end-user and a wide range of application problems with certain degree of robustness (impact of metamodel error). For problems with dynamical models, the typical approach is to use the batch updating method since that leads to a common broader Bayesian inference framework. 
    \item Design-under-uncertainty. Though this is not a common option, some software programs do offer the ability to perform some form of optimization under uncertainty and are most applicable to RBDO and RDO problems. Integrating state-of-the-art MCS techniques in this setting remains also a challenge. Implementations are typically computationally expensive or rely on approximate approaches for the uncertainty propagation.
\end{itemize}

\noindent Out of the different UQ software programs that exist, the following are worth direct mention as representing the state-of-the-art: 
\newline

\noindent\textbf{DAKOTA (https://dakota.sandia.gov/)} \\Developed by the Sandia National Laboratory and written in C++, DAKOTA is widely considered as the standard for UQ software and delivers both state-of-the-art research and robust, usable tools for optimization and UQ (Adams et al., 2009). It has a range of algorithms for all aforementioned UQ tasks and has a wide community that supports its continuous development. 
\newline

\noindent\textbf{OpenTURNS (http://www.openturns.org/)} \\It is an open-source (C++/Python library) initiative for the treatment of uncertainties and risk in simulations (Andrianov et al., 2007). It addresses all aforementioned UQ tasks apart from design under uncertainty. 
\newline

\noindent\textbf{UQLab (https://www.uqlab.com/)} \\Developed at ETH Zürich, UQLAB is a MATLAB-based general purpose UQ framework (Marelli and Sudret, 2014). Like OpenTURNS it addresses all the aforementioned UQ tasks apart from design under uncertainty. 
\newline

\noindent\textbf{OpenCossan (http://www.cossan.co.uk/software/open-cossan-engine.php)} \\It is the MATLAB based open-source version of the commercial software COSSAN-X (Patelli et al., 2017), which was initially developed to integrate UQ and reliability techniques within FEM analysis, with modules that extend across all aforementioned UQ tasks.

\noindent\textbf{UQpy  (http://muq.mit.edu/)} \\It is an continuously expanding open source (Python based) library of tools for UQ analysis (Olivier et al. 2020). It addresses all UQ tasks apart from design-under-uncertainty (at least at its current form).

\noindent\textbf{MUQ (http://www.cossan.co.uk/software/open-cossan-engine.php)} \\It is an open source (C++/Python library) collection of tools for UQ analysis, with emphasis on probabilistic modeling, Bayesian inference and surrogate modeling. Developments in MUQ have been slightly more accelerated recently, though overall it has not followed the same pace as other open-source initiatives mentioned above.

Beyond these specific software or open source tool collections, there is an increasing number of Python based open-source libraries that are offered by researchers for UQ analysis, for example UQ-Pyl (http://www.uq-pyl.com/), FilterPy (https://github.com/rlabbe/filterpy), Surrogate Modeling Toolbox (https://github.com/SMTorg/SMT). Most of them are focusing on specialized UQ tasks (or groups of tasks) and do not intend to establish a generalized UQ analysis workflow. The degree of ongoing improvement and bug fixes also varies substantially between these efforts. 

\section{References}
\label{sec:UQRef}

% Abbiati, G., Abdallah, I., Marelli, S., Sudret, B., and Stojadinovic, B. (2017)."Hierarchical Kriging Surrogate of the Seismic Response of a Steel Piping Network Based on Multi-Fidelity Hybrid and Computational Simulators." 7th International Conference on Advances in Experimental Structural Engineering (7AESE).
% Adams, B. M., Bohnhoff, W., Dalbey, K., Eddy, J., Eldred, M., Gay, D., Haskell, K., Hough, P. D., and Swiler, L. P. (2009). "DAKOTA, a multilevel parallel object-oriented framework for design optimization, parameter estimation, uncertainty quantification, and sensitivity analysis: version 5.0 user’s manual." Sandia National Laboratories, Tech. Rep. SAND2010-2183.
% Akiyama, M., Frangopol, D. M., and Ishibashi, H. (2020). "Toward life-cycle reliability-, risk-and resilience-based design and assessment of bridges and bridge networks under independent and interacting hazards: emphasis on earthquake, tsunami and corrosion." Structure and Infrastructure Engineering, 16(1), 26-50.
% Altieri, D., Tubaldi, E., De Angelis, M., Patelli, E., and Dall’Asta, A. (2018). "Reliability-based optimal design of nonlinear viscous dampers for the seismic protection of structural systems." Bulletin of Earthquake Engineering, 16(2), 963-982.
% Andrianov, G., Burriel, S., Cambier, S., Dutfoy, A., Dutka-Malen, I., De Rocquigny, E., Sudret, B., Benjamin, P., Lebrun, R., and Mangeant, F. (2007)."Open TURNS, an open source initiative to Treat Uncertainties, Risks’ N Statistics in a structured industrial approach." Proceedings of the ESREL’2007 Safety and Reliability Conference, Stavenger: Norway.
% Angelikopoulos, P., Papadimitriou, C., and Koumoutsakos, P. (2015). "X-TMCMC: Adaptive kriging for Bayesian inverse modeling." Computer Methods in Applied Mechanics and Engineering, 289, 409-428.
% Astroza, R., Ebrahimian, H., and Conte, J. P. (2017). "Batch and Recursive Bayesian Estimation Methods for Nonlinear Structural System Identification." Risk and Reliability Analysis: Theory and Applications, Springer, 341-364.
% Au, S. K., and Beck, J. L. (2001). "Estimation of small failure probabilities in high dimensions by subset simulation." Probabilistic Engineering Mechanics, 16(4), 263-277.
% Au, S. K., and Beck, J. L. (2003a). "Importance sampling in high dimensions." Structural Safety, 25(2), 139-163.
% Au, S. K., and Beck, J. L. (2003b). "Subset simulation and its applications to seismic risk based on dynamic analysis." Journal of Engineering Mechanics, ASCE, 129(8), 901-917.
% Baker, J. W., and Cornell, C. A. (2005). "A vector-valued ground motion intensity measure consisting of spectral acceleration and epsilon." Earthquake Engineering and Structural Dynamics, 34(10), 1193-1217.
% Baker, J. W., and Cornell, C. A. (2008). "Uncertainty propagation in probabilistic seismic loss estimation." Structural Safety, 30(3), 236-252.
% Balesdent, M., Morio, J., and Marzat, J. (2013). "Kriging-based adaptive importance sampling algorithms for rare event estimation." Structural Safety, 44, 1-10.
% Bansal, S., and Cheung, S. H. (2018). "A subset simulation based approach with modified conditional sampling and estimator for loss exceedance curve computation." Reliability Engineering & System Safety, 177, 94-107.
% Barbato, M., Petrini, F., Unnikrishnan, V. U., and Ciampoli, M. (2013). "Performance-based hurricane engineering (PBHE) framework." Structural Safety, 45, 24-35.
% Beck, J. L. (2010). "Bayesian system identification based on probability logic." Structural Control and Health Monitoring, 17(7), 825-847.
% Beck, J. L., and Taflanidis, A. (2013). "Prior and posterior robust stochastic predictions for dynamical systems using probability logic." Journal of Uncertainty Quantification, 3(4), 271-288.
% Bect, J., Li, L., and Vazquez, E. (2017). "Bayesian subset simulation." SIAM/ASA Journal on Uncertainty Quantification, 5(1), 762-786.
% Beer, M., Ferson, S., and Kreinovich, V. (2013). "Imprecise probabilities in engineering analyses." Mechanical systems and signal processing, 37(1-2), 4-29.
% Benowitz, B. A., and Deodatis, G. (2015). "Simulation of wind velocities on long span structures: A novel stochastic wave based model." Journal of wind engineering and industrial aerodynamics, 147, 154-163.
% Bernier, C., and Padgett, J. E. (2019). "Fragility and risk assessment of aboveground storage tanks subjected to concurrent surge, wave, and wind loads." Reliability Engineering & System Safety, 191, 106571.
% Betz, W., Papaioannou, I., and Straub, D. (2016). "Transitional Markov chain Monte Carlo: observations and improvements." Journal of Engineering Mechanics, 142(5), 04016016.
% Beyer, H.-G., and Sendhoff, B. (2007). "Robust optimization–a comprehensive survey." Computer methods in applied mechanics and engineering, 196(33-34), 3190-3218.
% Bichon, B. J., Eldred, M. S., Mahadevan, S., and McFarland, J. M. (2013). "Efficient global surrogate modeling for reliability-based design optimization." Journal of Mechanical Design, 135(1), 011009.
% Bijelić, N., Lin, T., and Deierlein, G. G. (2018). "Validation of the SCEC Broadband Platform simulations for tall building risk assessments considering spectral shape and duration of the ground motion." Earthquake Engineering & Structural Dynamics, 47(11), 2233-2251.
% Bobby, S., Suksuwan, A., Spence, S. M., and Kareem, A. (2017). "Reliability-based topology optimization of uncertain building systems subject to stochastic excitation." Structural Safety, 66, 1-16.
% Bourinet, J.-M. (2016). "Rare-event probability estimation with adaptive support vector regression surrogates." Reliability Engineering & System Safety, 150, 210-221.
% Bourinet, J., Mattrand, C., and Dubourg, V. (2009)."A review of recent features and improvements added to FERUM software." Proc. of the 10th International Conference on Structural Safety and Reliability (ICOSSAR’09).
% Catanach, T. A., and Beck, J. L. (2018). "Bayesian Updating and Uncertainty Quantification using Sequential Tempered MCMC with the Rank-One Modified Metropolis Algorithm." arXiv preprint arXiv:1804.08738.
% Cha, E. J., and Ellingwood, B. R. (2012). "Risk-averse decision-making for civil infrastructure exposed to low-probability, high-consequence events." Reliability Engineering & System Safety, 104, 27-35.
% Chatzi, E. N., and Smyth, A. W. (2009). "The unscented Kalman filter and particle filter methods for nonlinear structural system identification with non‐collocated heterogeneous sensing." Structural Control and Health Monitoring: The Official Journal of the International Association for Structural Control and Monitoring and of the European Association for the Control of Structures, 16(1), 99-123.
% Ching, J., and Chen, Y.-C. (2007). "Transitional Markov chain Monte Carlo method for Bayesian model updating, model class selection, and model averaging." Journal of Engineering Mechanics, 133(7), 816-832.
% Chun, J., Song, J., and Paulino, G. H. (2019). "System-reliability-based design and topology optimization of structures under constraints on first-passage probability." Structural Safety, 76, 81-94.
% Ciampoli, M., Petrini, F., and Augusti, G. (2011). "Performance-based wind engineering: towards a general procedure." Structural Safety, 33(6), 367-378.
% Clement, W., Gilles, D., Bertrand, I., and Vincent, M. (2018). "Methods in Structural Reliaiblity, https://rdrr.io/github/clemlaflemme/mistral/f/."
% de Baar, J., Roberts, S., Dwight, R., and Mallol, B. (2015). "Uncertainty quantification for a sailing yacht hull, using multi-fidelity kriging." Computers & Fluids, 123, 185-201.
% Deb, A., Zha, A. L., Caamaño-Withall, Z. A., Conte, J. P., and Restrepo, J. I. (2019a)."Towards a simplified and rigorous performance-based seismic design of ordinary standard bridges in California." 7th ECCOMAS Thematic Conference on Computational Methods in Structural Dynamics and Earthquake Engineering (COMPDYN 2019), 24-26 June, Crete, Greece.
% Deb, A., Zha, A. L., Caamaño-Withall, Z. A., Conte, J. P., and Restrepo, J. I. (2019b)."Parametric Probabilistic Seismic Performance Assessment Framework for Ordinary Standard Bridges." 3th International Conference on Applications of Statistics and Probability in Civil Engineering (ICASP13), May 26-30, Seoul, South Korea.
% Deodatis, G., and Shinozuka, M. (1988). "Auto-regressive model for nonstationary stochastic processes." Journal of engineering mechanics, 114(11), 1995-2012.
% Ding, F., and Kareem, A. (2018). "A multi-fidelity shape optimization via surrogate modeling for civil structures." Journal of Wind Engineering and Industrial Aerodynamics, 178, 49-56.
% dos Santos, K. R., Kougioumtzoglou, I. A., and Beck, A. T. (2016). "Incremental dynamic analysis: A nonlinear stochastic dynamics perspective." Journal of Engineering Mechanics, 142(10), 06016007.
% Dubourg, V., Sudret, B., and Bourinet, J.-M. (2011). "Reliability-based design optimization using kriging surrogates and subset simulation." Structural and Multidisciplinary Optimization, 44(5), 673-690.
% Echard, B., Gayton, N., and Lemaire, M. (2011). "AK-MCS: An active learning reliability method combining Kriging and Monte Carlo Simulation." Structural Safety, 33(2), 145-154.
% Erazo, K., and Nagarajaiah, S. (2018). "Bayesian structural identification of a hysteretic negative stiffness earthquake protection system using unscented Kalman filtering." Structural Control and Health Monitoring, 25(9), e2203.
% Esposito, S., Iervolino, I., d'Onofrio, A., Santo, A., Cavalieri, F., and Franchin, P. (2015). "Simulation‐based seismic risk assessment of gas distribution networks." Computer‐Aided Civil and Infrastructure Engineering, 30(7), 508-523.
% Fenton, G. A., and Vanmarcke, E. H. (1990). "Simulation of random fields via local average subdivision." Journal of Engineering Mechanics, 116(8), 1733-1749.
% Fischer, E. C., Varma, A. H., and Agarwal, A. (2019). "Performance-Based Structural Fire Engineering of Steel Building Structures: Design-Basis Compartment Fires." Journal of Structural Engineering, 145(9), 04019090.
% Flint, M., Dhulipala, L., Shahtaheri, Y., Tahir, H., Ladipo, T., Eatherton, M., Irish, J., Olgun, C., Reichard, G., and Rodriguez-Marek, A. (2016)."Developing a decision framework for multi-hazard design of resilient, sustainable buildings." 1st Int. Conf. on Natural Hazards and Infrastructure (ICONHIC2016), 28-30 June, Chania, Greece.
% Gentile, R., and Galasso, C. (2020). "Gaussian process regression for seismic fragility assessment of building portfolios." Structural Safety, 87, 101980.
% Geraci, G., Eldred, M. S., and Iaccarino, G. (2017)."A multifidelity multilevel Monte Carlo method for uncertainty propagation in aerospace applications." 19th AIAA Non-Deterministic Approaches Conference, 1951.
% Ghanem, R., Higdon, D., and Owhadi, H. (2017). Handbook of uncertainty quantification, Springer.
% Ghanem, R. G., and Spanos, P. D. (1991). Stochastic finite elements: a spectral approach, Springer.
% Gidaris, I., Taflanidis, A. A., and Mavroeidis, G. P. (2014)."Surrogate modeling implementation for assessment of seismic risk utilizing stochastic ground motion modeling." 2nd European Conference in Earthquake Engineering and Seismology, August 25-29, Istanbul, Turkey.
% Gidaris, I., Taflanidis, A. A., and Mavroeidis, G. P. (2015). "Kriging metamodeling in seismic risk assessment based on stochastic ground motion models." Earthquake Engineering & Structural Dynamics, 44(14), 2377–2399.
% Gidaris, I., Taflanidis, A. A., and Mavroeidis, G. P. (2017). "Multiobjective Design of Supplemental Seismic Protective Devices Utilizing Lifecycle Performance Criteria." Journal of Structural Engineering, 144(3), 04017225.
% Giovanis, D. G., Papaioannou, I., Straub, D., and Papadopoulos, V. (2017). "Bayesian updating with subset simulation using artificial neural networks." Computer Methods in Applied Mechanics and Engineering, 319, 124-145.
% Gorissen, D., Couckuyt, I., Demeester, P., Dhaene, T., and Crombecq, K. (2010). "A surrogate modeling and adaptive sampling toolbox for computer based design." Journal of Machine Learning Research, 11(Jul), 2051-2055.
% Goulet, C. A., Haselton, C. B., Mitrani-Reiser, J., Beck, J. L., Deierlein, G., Porter, K. A., and Stewart, J. P. (2007). "Evaluation of the seismic performance of code-conforming reinforced-concrete frame building-From seismic hazard to collapse safety and economic losses." Earthquake Engineering and Structural Dynamics, 36(13), 1973-1997.
% Greco, R., Lucchini, A., and Marano, G. C. (2015). "Robust design of tuned mass dampers installed on multi-degree-of-freedom structures subjected to seismic action." Engineering Optimization, 47(8), 1009-1030.
% Gu, Q., Conte, J. P., Elgamal, A., and Yang, Z. (2009). "Finite element response sensitivity analysis of multi-yield-surface J2 plasticity model by direct differentiation method." Computer methods in applied mechanics and engineering, 198(30-32), 2272-2285.
% Haukaas, T., and Der Kiureghian, A. (2007). "Methods and object-oriented software for FE reliability and sensitivity analysis with application to a bridge structure." Journal of Computing in Civil Engineering, 21(3), 151-163.
% Haukaas, T., and Mahsuli, M. (2012)."Reliability-based design optimization with uncertain cost." IFIP WG 75 Working Conference on Reliability and Optimization of Structural Systems, Yerevan, Armenia.
% Hu, Z., and Mahadevan, S. (2019). "Probability models for data-driven global sensitivity analysis." Reliability Engineering & System Safety, 187, 40-57.
% Huang, M., Li, Q., Chan, C. M., Lou, W., Kwok, K. C., and Li, G. (2015). "Performance-based design optimization of tall concrete framed structures subject to wind excitations." Journal of Wind Engineering and Industrial Aerodynamics, 139, 70-81.
% Jayaram, N., and Baker, J. W. (2010). "Efficient sampling and data reduction techniques for probabilistic seismic lifeline risk assessment." Earthquake Engineering & Structural Dynamics, 39(10), 1109-1131.
% Jia, G., and Taflanidis, A. A. (2013). "Kriging metamodeling for approximation of high-dimensional wave and surge responses in real-time storm/hurricane risk assessment." Computer Methods in Applied Mechanics and Engineering, 261-262, 24-38.
% Jia, G., Taflanidis, A., and Beck, J. L. (2014)."Adaptive stochastic sampling using kernel density approximations." 7th International Conference on Computational Stochastic Mechanics June 15-18, Santorini, Greece.
% Jia, G., Taflanidis, A. A., and Beck, J. L. (2017). "A new adaptive rejection sampling method using kernel density approximations and its application to Subset Simulation." ASCE-ASME Journal of Risk and Uncertainty in Engineering Systems, Part A: Civil Engineering, 3(2), D4015001.
% Jia, G., and Gardoni, P. (2018). "State-dependent stochastic models: A general stochastic framework for modeling deteriorating engineering systems considering multiple deterioration processes and their interactions." Structural Safety, 72, 99-110.
% Kennedy, M. C., and O'Hagan, A. (2001). "Bayesian calibration of computer models." Journal of the Royal Statistical Society: Series B (Statistical Methodology), 63(3), 425-464.
% Kleijnen, J. P. C. (2008). Design and analysis of simulation experiments, Springer.
% Koduru, S., and Haukaas, T. (2010). "Feasibility of FORM in finite element reliability analysis." Structural Safety, 32(2), 145-153.
% Kohrangi, M., Vamvatsikos, D., and Bazzurro, P. (2016). "Implications of intensity measure selection for seismic loss assessment of 3-D buildings." Earthquake Spectra, 32(4), 2167-2189.
% Kontoroupi, T., and Smyth, A. W. (2017). "Online Bayesian model assessment using nonlinear filters." Structural Control and Health Monitoring, 24(3), e1880.
% Kyprioti, A. P., Zhang, J., and Taflanidis, A. A. (2020). "Adaptive design of experiments for global Kriging metamodeling through cross-validation information." Structural and Multidisciplinary Optimization, 1-23.
% Le, V., and Caracoglia, L. (2020). "A neural network surrogate model for the performance assessment of a vertical structure subjected to non-stationary, tornadic wind loads." Computers & Structures, 231, 106208.
% Li, C., and Mahadevan, S. (2016). "An efficient modularized sample-based method to estimate the first-order Sobol׳ index." Reliability Engineering & System Safety, 153, 110-121.
% Li, D.-Q., Yang, Z.-Y., Cao, Z.-J., Au, S.-K., and Phoon, K.-K. (2017). "System reliability analysis of slope stability using generalized subset simulation." Applied Mathematical Modelling, 46, 650-664.
% Li, J., and Chen, J. (2009). Stochastic dynamics of structures, John Wiley & Sons.
% Li, J., Li, J., and Xiu, D. (2011). "An efficient surrogate-based method for computing rare failure probability." Journal of Computational Physics, 230(24), 8683-8697.
% Li, Y., and Kareem, A. (1991). "Simulation of multivariate nonstationary random processes by FFT." Journal of Engineering Mechanics, 117(5), 1037-1058.
% Li, Y., and Conte, J. P. (2018). "Probabilistic performance‐based optimum design of seismic isolation for a California high‐speed rail prototype bridge." Earthquake Engineering & Structural Dynamics, 47(2), 497-514.
% Liu, H., Ong, Y.-S., and Cai, J. (2018). "A survey of adaptive sampling for global metamodeling in support of simulation-based complex engineering design." Structural and Multidisciplinary Optimization, 57(1), 393-416.
% Lophaven, S. N., Nielsen, H. B., and Søndergaard, J. (2002). "DACE-A Matlab Kriging toolbox, version 2.0." Technical University of Denmark.
% Mantoglou, A., and Wilson, J. L. (1982). "The turning bands method for simulation of random fields using line generation by a spectral method." Water Resources Research, 18(5), 1379-1394.
% Marelli, S., and Sudret, B. (2014). "UQLab: A framework for uncertainty quantification in Matlab." Vulnerability, Uncertainty, and Risk: Quantification, Mitigation, and Management, 2554-2563.
% McGuire, R. K. (2004). Seismic hazard and risk analysis, Earthquake engineering research institute.
% Medina, J. C., and Taflanidis, A. (2014). "Adaptive importance sampling for optimization under uncertainty problems." Computer Methods in Applied Mechanics and Engineering, 279, 133–162.
% Mehmani, A., Chowdhury, S., Meinrenken, C., and Messac, A. (2018). "Concurrent surrogate model selection (COSMOS): optimizing model type, kernel function, and hyper-parameters." Structural and Multidisciplinary Optimization, 57(3), 1093-1114.
% Murphy, K. P. (2012). "Machine learning: a probabilistic perspective (adaptive computation and machine learning series)." Mit Press. ISBN, 621485037, 15.
% Muto, M., and Beck, J. L. (2008). "Bayesian updating and model class selection for hysteretic structural models using stochastic simulation." Journal of Vibration and Control, 14(1-2), 7-34.
% Oh, C. K., and Beck, J. L. (2018). "A Bayesian Learning Method for Structural Damage Assessment of Phase I IASC-ASCE Benchmark Problem." KSCE Journal of Civil Engineering, 22(3), 987-992.
% Olivier, A., and Smyth, A. W. (2017). "Particle filtering and marginalization for parameter identification in structural systems." Structural Control and Health Monitoring, 24(3), e1874.
% Olivier, A., Giovanis, D., Aakash, B., Chauhan, M., Vandanapu, L., and Shields, M. D. (2020). "UQpy: A general purpose Python package and development environment for uncertainty quantification." Journal of Computational Science, 47, 101204.
% Papadimitriou, D., Panagiotopoulos, D., and Mourelatos, Z. (2018)."Reliability Based Design Optimization Using First, Second and Quasi-Second Order Saddlepoint Approximations." 2018 AIAA Non-Deterministic Approaches Conference, 2172.
% Papaioannou, I., Breitung, K., and Straub, D. (2018). "Reliability sensitivity estimation with sequential importance sampling." Structural Safety, 75, 24-34.
% Patelli, E., Broggi, M., Tolo, S., and Sadeghi, J. (2017)."Cossan software: A multidisciplinary and collaborative software for uncertainty quantification." Proceedings of the 2nd ECCOMAS thematic conference on uncertainty quantification in computational sciences and engineering, UNCECOMP.
% Peherstorfer, B., Willcox, K., and Gunzburger, M. (2018). "Survey of multifidelity methods in uncertainty propagation, inference, and optimization." SIAM Review, 60(3), 550-591.
% Picheny, V., Ginsbourger, D., Roustant, O., Haftka, R. T., and Kim, N. H. (2010). "Adaptive designs of experiments for accurate approximation of a target region." Journal of Mechanical Design, 132(7), 071008.
% Porter, K. A., Beck, J. L., and Shaikhutdinov, R. V. (2002). "Sensitivity of building loss estimates to major uncertain variables." Earthquake Spectra, 18(4), 719-743.
% Quiroz, M., Tran, M.-N., Villani, M., and Kohn, R. (2018). "Speeding up MCMC by delayed acceptance and data subsampling." Journal of Computational and Graphical Statistics, 27(1), 12-22.
% Rahman, S. (2016). "The f-sensitivity index." SIAM/ASA Journal on Uncertainty Quantification, 4(1), 130-162.
% Ramancha, M. K., Madarshahian, R., Astroza, R., and Conte, J. P. (2020). Non-unique Estimates in Material Parameter Identification of Nonlinear FE Models Governed by Multiaxial Material Models Using Unscented Kalman Filtering, Springer.
% Ramancha, M. K., Astroza, R., Conte, J. P., Restrepo, J. I., and Todd, M. D. (2021a)."Bayesian nonlinear finite element model updating of a full-scale bridge column using sequential Monte Carlo." 38th International Modal Analysis Conference (IMAC XXXVIII), February 10-13, Houston, TX,.
% Ramancha, M. K., Astroza, R., Madarshahian, R., and Conte, J. P. (2021b). "Bayesian updating and identifiability assessment of nonlinear finite element model." Mechanical Systems and Signal Processing.
% Resio, D. T., Boc, S. J., Borgman, L., Cardone, V., Cox, A., Dally, W. R., Dean, R. G., Divoky, D., Hirsh, E., Irish, J. L., Levinson, D., Niedoroda, A., Powell, M. D., Ratcliff, J. J., Stutts, V., Suhada, J., Toro, G. R., and Vickery, P. J. (2007). "White paper on estimating hurricane inundation probabilities."
% Riggs, H., Robertson, I. N., Cheung, K. F., Pawlak, G., Young, Y. L., and Yim, S. C. (2008)."Experimental simulation of tsunami hazards to buildings and bridges." dalam Proceedings of 2008 NSF Engineering Research and Innovation Conference, Knoxville, Tennessee.
% Saltelli, A. (2002). "Making best use of model evaluations to compute sensitivity indices." Computer physics communications, 145(2), 280-297.
% Schuëller, G. I., Pradlwater, H. J., and Koutsourelakis, P. S. (2004). "A critical appraisal of reliability estimation procedures for high dimensions." Probabilistic Engineering Mechanics, 19(4), 463-474.
% Shields, M., Deodatis, G., and Bocchini, P. (2011). "A simple and efficient methodology to approximate a general non-Gaussian stationary stochastic process by a translation process." Probabilistic Engineering Mechanics, 26(4), 511-519.
% Shin, H., and Singh, M. P. (2014). "Minimum failure cost-based energy dissipation system designs for buildings in three seismic regions Part II: Application to viscous dampers." Engineering Structures, DOI: 10.1016/j.engstruct.2014.05.012.
% Shinozuka, M., and Deodatis, G. (1991). "Simulation of stochastic processes by spectral representation."
% Smith, M. A., and Caracoglia, L. (2011). "A Monte Carlo based method for the dynamic “fragility analysis” of tall buildings under turbulent wind loading." Engineering Structures, 33(2), 410-420.
% Smith, R. C. (2013). Uncertainty quantification: theory, implementation, and applications, Siam.
% Sobol', I. y. M. (1990). "On sensitivity estimation for nonlinear mathematical models." Matematicheskoe Modelirovanie, 2(1), 112-118.
% Spall, J. C. (2003). Introduction to stochastic search and optimization, Wiley-Interscience, New York.
% Spanos, P. T. (1983). "ARMA algorithms for ocean wave modeling."
% Spence, S. M., and Gioffrè, M. (2012). "Large scale reliability-based design optimization of wind excited tall buildings." Probabilistic Engineering Mechanics, 28, 206-215.
% Stern, R. E., Song, J., and Work, D. B. (2017). "Accelerated Monte Carlo system reliability analysis through machine-learning-based surrogate models of network connectivity." Reliability Engineering & System Safety, 164, 1-9.
% Su, L., Wan, H.-P., Dong, Y., Frangopol, D. M., and Ling, X.-Z. (2018). "Efficient Uncertainty Quantification of Wharf Structures under Seismic Scenarios Using Gaussian Process Surrogate Model." Journal of Earthquake Engineering, 1-22.
% Sudret, B. (2008). "Global sensitivity analysis using polynomial chaos expansions." Reliability engineering & system safety, 93(7), 964-979.
% Suksuwan, A., and Spence, S. M. (2018). "Optimization of uncertain structures subject to stochastic wind loads under system-level first excursion constraints: A data-driven approach." Computers & Structures, 210, 58-68.
% Taflanidis, A. A., and Jia, G. (2011). "A simulation-based framework for risk assessment and probabilistic sensitivity analysis of base-isolated structures." Earthquake Engineering & Structural Dynamics, 40, 1629–1651.
% Tripathy, R., and Bilionis, I. (2018). "Deep UQ: Learning deep neural network surrogate models for high dimensional uncertainty quantification." arXiv preprint arXiv:1802.00850.
% Vamvatsikos, D. (2013). "Derivation of new SAC/FEMA performance evaluation solutions with second‐order hazard approximation." Earthquake Engineering & Structural Dynamics, 42(8), 1171-1188.
% Vamvatsikos, D. (2014). "Seismic performance uncertainty estimation via IDA with progressive accelerogram-wise latin hypercube sampling." Journal of Structural Engineering, 140(8), A4014015.
% Vetter, C., and Taflanidis, A. A. (2012). "Global sensitivity analysis for stochastic ground motion modeling in seismic-risk assessment." Soil Dynamics and Earthquake Engineering, 38, 128–143.
% Vlachos, C., Papakonstantinou, K. G., and Deodatis, G. (2018). "Predictive model for site specific simulation of ground motions based on earthquake scenarios." Earthquake Engineering & Structural Dynamics, 47(1), 195-218.
% Wang, L., McCullough, M., and Kareem, A. (2014). "Modeling and simulation of nonstationary processes utilizing wavelet and Hilbert transforms." Journal of Engineering Mechanics, 140(2), 345-360.
% Wang, Z., and Der Kiureghian, A. (2016). "Tail-equivalent linearization of inelastic multisupport structures subjected to spatially varying stochastic ground motion." Journal of Engineering Mechanics, 142(8), 04016053.
% Wang, Z., and Song, J. (2016). "Cross-entropy-based adaptive importance sampling using von Mises-Fisher mixture for high dimensional reliability analysis." Structural Safety, 59, 42-52.
% Wang, Z., Zentner, I., and Zio, E. (2018). "A Bayesian framework for estimating fragility curves based on seismic damage data and numerical simulations by adaptive neural networks." Nuclear Engineering and Design, 338, 232-246.
% Wang, Z., and Shafieezadeh, A. (2019)."Reliability-based Bayesian updating using machine learning." 3th International Conference on Applications of Statistics and Probability in Civil Engineering, ICASP13, May 26-30, Seoul, South Korea.
% Wei, Z., Tao, T., ZhuoShu, D., and Zio, E. (2013). "A dynamic particle filter-support vector regression method for reliability prediction." Reliability Engineering & System Safety, 119, 109-116.
% Whittaker, A., Hamburger, R., and Mahoney, M. (2003)."Performance-based engineering of buildings and infrastructure for extreme loadings." Proceedings, AISC-SINY Symposium on Resisting Blast and Progressive Collapse, American Institute of Steel Construction, New York.
% Zeldin, B., and Spanos, P. (1996). "Random field representation and synthesis using wavelet bases."
% Zhang, J., and Taflanidis, A. (2018a). "Multi-objective optimization for design under uncertainty problems through surrogate modeling in augmented input space." Structural and Multidisciplinary Optimization, 1-22.
% Zhang, J., and Taflanidis, A. A. (2018b). "Adaptive Kriging Stochastic Sampling and Density Approximation and Its Application to Rare-Event Estimation." ASCE-ASME Journal of Risk and Uncertainty in Engineering Systems, Part A: Civil Engineering, 4(3), 04018021.
% Zhang, J., and Taflanidis, A. A. (2019). "Accelerating MCMC via Kriging-based adaptive independent proposals and delayed rejection." Computer Methods in Applied Mechanics and Engineering, 355, 1124-1147.
% Zhang, R., Liu, Y., and Sun, H. (2020). "Physics-guided convolutional neural network (PhyCNN) for data-driven seismic response modeling." Engineering Structures, 215, 110704.
% Zhou, Q., Shao, X., Jiang, P., Gao, Z., Zhou, H., and Shu, L. (2016). "An active learning variable-fidelity metamodelling approach based on ensemble of metamodels and objective-oriented sequential sampling." Journal of Engineering Design, 27(4-6), 205-231.
% Zhu, M., Yang, Y., Guest, J. K., and Shields, M. D. (2017). "Topology optimization for linear stationary stochastic dynamics: applications to frame structures." Structural Safety, 67, 116-131.

