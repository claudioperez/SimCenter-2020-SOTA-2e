
\begin{partbacktext}
\part{Cross-Cutting Methodologies}
\label{part:cross}

This section of the report examines two related cross-cutting topics---uncertainty quantification and artificial intelligence---which have applications across all phases of natural hazards engineering (NHE).  Both areas are developing rapidly, propelled by the capabilities of modern high-performance computing (HPC), information technologies (IT), and data harvesting technologies, and supported by algorithmic developments.  

Uncertainty quantification (UQ) covers a broad range of topics that include: (1) characterization of uncertainties in natural hazards, their damaging effects on the built environment, and the resulting consequences on communities; (2) propagation of uncertainties through simulations of natural hazards through to their consequences; (3) statistical calibration of simulation models, including Bayesian inference methods to update models with observed data; and (4) design under uncertainty, including design of physical assets through to strategies and practices to mitigate risk and promote recovery from natural disasters.  

Artificial intelligence (AI) and machine learning (ML) have a wide range of potential applications to NHE, the capabilities of which are just beginning to be realized, including:

\begin{enumerate}
    \item Developing features of buildings and other assets that form the basis for simulating the effects of natural hazards on communities;
    \vspace{2mm}
    \item Developing surrogate data-driven models at various scales, ranging from material models in finite-element simulations, to models of natural hazard intensity parameters, to simplified models of buildings and other assets in regional simulations; and
    \vspace{2mm}
    \item Detecting and assimilating observational data from post-disaster reconnaissance.
\end{enumerate}

As described in Chapter \ref{chapter:cc_aiml}, AI/ML tools encompass a range of techniques, including knowledge-based expert systems, statistical-based neural networks, kernel-based methods, and deep-learning approaches.
While the AI/ML show great potential and promise to revolutionize NHE, success in this regard hinges on identifying appropriate uses of these technologies and taking care to validate and develop confidence in the methods.


\end{partbacktext}