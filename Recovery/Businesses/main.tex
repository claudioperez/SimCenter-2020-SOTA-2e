%%%%%%%%%%%%%%%%%%%% author.tex %%%%%%%%%%%%%%%%%%%%%%%%%%%%%%%%%%%
%
% sample root file for your "contribution" to a contributed volume
%
% Use this file as a template for your own input.
%
%%%%%%%%%%%%%%%% Springer %%%%%%%%%%%%%%%%%%%%%%%%%%%%%%%%%%%%%%%%%


%% RECOMMENDED %%%%%%%%%%%%%%%%%%%%%%%%%%%%%%%%%%%%%%%%%%%%%%%%%%%
%\documentclass[graybox]{svmult}
%
%% choose options for [] as required from the list
%% in the Reference Guide
%
%\usepackage{mathptmx}       % selects Times Roman as basic font
%\usepackage{helvet}         % selects Helvetica as sans-serif font
%\usepackage{courier}        % selects Courier as typewriter font
%\usepackage{type1cm}        % activate if the above 3 fonts are
                             % not available on your system
%
%\usepackage{makeidx}         % allows index generation
%\usepackage{graphicx}        % standard LaTeX graphics tool
%                             % when including figure files
%\usepackage{multicol}        % used for the two-column index
%\usepackage[bottom]{footmisc}% places footnotes at page bottom
%
%% see the list of further useful packages
%% in the Reference Guide
%
%\makeindex             % used for the subject index
%                       % please use the style svind.ist with
%                       % your makeindex program
%
%%%%%%%%%%%%%%%%%%%%%%%%%%%%%%%%%%%%%%%%%%%%%%%%%%%%%%%%%%%%%%%%%%%%%%%%%%%%%%%%%%%%%%%%%%
%
%\begin{document}

\title{Businesses}
% Use \titlerunning{Short Title} for an abbreviated version of
% your contribution title if the original one is too long
\author{
    \textbf{Rodrigo Costa}}
\tocauthor{}
\authorrunning{Costa}
% Use \authorrunning{Short Title} for an abbreviated version of
% your contribution title if the original one is too long
%\institute{Name of First Author \at Name, Address of Institute, %\email{name@email.address}
%\and Name of Second Author \at Name, Address of Institute %\email{name@email.address}}
%
% Use the package "url.sty" to avoid
% problems with special characters
% used in your e-mail or web address
%
\maketitle
Businesses play a major role in the recovery of a community because they provide a large portion of the local jobs. Without employment, households may experience difficulty in financing the restoration of their homes and will be forced to relocate \citep{bolin1983recovery,wang2015influencing}. With fewer clients, even businesses that survive a disaster may be forced to close, creating a synergy between slow business recovery and failed housing recovery. Despite its importance for community recovery, the literature on post-disaster business recovery is scarce. On the one hand, the rarity of extreme natural events make the systematic and longitudinal collection business recovery data challenging. On the other hand, larger economic cycles exert a strong influence on the well-being of individual firms, making it difficult to disaggregate macroeconomic and disaster-related effects.

Some of the most comprehensive studies of business recovery in the U.S. were conducted by the Disaster Research Center \citep{webb2000businesses}. These studies found that direct physical damage is only one of the many factors influencing business loss and recovery. Often, physical damage plays a secondary role to financial and market stability.\ 

%--------------------------
\section{Input and Output Data} 
%--------------------------
The majority of the existing business recovery models have been developed using survey data, collected shortly after or even years after a disaster. These surveys collected data on the number of full time employees, a surrogate measure of size, age, industry sector, and pre-disaster financial conditions of the surveyed businesses. Data about the business neighborhood and how it was impacted have also been shown to be relevant \citep{chang2010urban}. The level of engagement in disaster preparedness activities of the businesses is another dimension of interest. Measures of disaster impact, such as physical damage or disruptions to utilities are also important data. Finally, scholars have also argued that the ability of owners to make quick decisions about relocation, financial injection, and/or switching to new business models is crucial for recovery \citep{stevenson2014organizational,morrish2011entrepreneurial}. These data can also be obtained from surveys. The outputs of business recovery models are often a list of factors that lead businesses to recover more quickly or in a more sustainable manner.\

%--------------------------
\section{Concepts in Business Recovery Modeling} 
%--------------------------
Businesses are diverse in nature. Industry sector, size, age, being a franchise, location, primary market, and financial stability are a few of the factors that may influence how a business responds to a shock and recovers from it. For this reason, understanding what factors contributed to the successful recovery of business in past events is important to guide data collection and development of new models.\ 

Within the U.S., business recovery was studied in the aftermath of a few major events. Short-term business recovery after the 1994 Northridge, California, earthquake was affected by direct physical impact on business operations, ecological aspects, and neighborhood location of the business \citep{dahlhamer1998rebounding}. Conversely, in the long-term, the business sector, its primary market, disaster impacts, and the broader economic climate affected business recovery. The post-disaster business climate was also determinant for the recovery of businesses affected by the 1989 Loma Prieta, California, earthquake and the 1992 Hurricane Andrew \citep{webb2002predicting}. Business size and disaster preparedness were also significant predictors for recovery after these two events. In 2001, a meta study involving seven disaster in seven communities across the U.S. identified the business and market stability, asset losses, and entrepreneurial savvy as good predictors of recovery \citep{alesch2001organizations}. Also in 2001, after the Nisqually, Washington, earthquake, size, ownership over property, access to resources, market diversification, market stability, and neighborhood conditions influenced the recovery of businesses in two districts of Seattle. \citep{chang2002disaster}. \

Outside of the U.S., the most extensively studied event in terms of business recovery is the 2010 and 2011 Canterbury earthquake sequence. These events led to the cordoning of the Christchurch Central Business District for nearly two and a half-years \citep{brown2019business}. One outcome was that many local businesses had to relocate to distant regions, thereby losing much-needed staff and customers \citep{morrish2020post}. Businesses that struggled to gain new customers and found new employees in their new location suffered the most, denoting a correlation between business relocation and recovery. Businesses whose suppliers were disrupted were significantly more likely to experience decreased productivity \citep{brown2019business}. The ability of owners to make quick decisions about relocation, financial injection, switching to new business models, or stating inter-organizational cooperation for sharing of knowledge, resources, risks, and profits was crucial for recovery \citep{stevenson2014organizational,morrish2011entrepreneurial}. Beyond that, commercial insurance, industry sector, property ownership, and customer issues were shown to affect recovery, whereas business size and age did not (\cite{brown2017efficacy,brown2015factors}; and \cite{kachali2015industry}). \

The majority of empirical studies focusing on business recovery have utilized statistical modeling techniques to identify various the predictors of successful recovery, but without focusing on the development of predictive models \citep{aghababaei2020quantifying}; among those, logistic regression was widely used  (\cite{watson2020importance,marshall2015predicting}; and \cite{dahlhamer1998rebounding}). Logistic regression models share many similarities with linear regression models. The important difference is that the independent variable, \textit{Y}, is binary, e.g., having states ``Yes'' and ``No'', or ``1'' and ``0''. Logistic regression is named after the function used at the core of the method, the logistic function. This is an S-shaped curve that can take any real-valued number and map it into a value between 0 and 1, but never exactly at those limits. The logistic function is used to estimate the probability that $Y=1$

\begin{equation}
    P(Y=1) = \frac{\exp(\beta_0 + \sum_{i=1}^{p}\beta_i \cdot x_i)}{1 + \exp(\beta_0 + \sum_{i=1}^{p}\beta_i \cdot x_i)}
    \label{eq:probability}
\end{equation}

\noindent which can be rearranged as

\begin{equation}
    \ln \Bigg(\frac{P(Y=1)}{1-P(Y=1)}\Bigg) = \beta_0 + \sum_{i=1}^{p}\beta_i \cdot x_i 
    \label{eq:logit}
\end{equation}

\noindent where \textit{p} is the number of predictors, and the coefficients $\beta_i$ can be estimated using maximum-likelihood estimation. Although logistic regression is a linear method, the transformation using the logistic function makes it so that we can no longer understand the predictions as a linear combination of the inputs. Everything else being equal, the coefficients $\beta_i$ represent the increase in the odds-ratio of $Y=1$ to $Y=0$ when $x_i$ is increased by one unit. Section 4.4 in \citet{friedman2001elements} provides in-depth explanations of logistic regression models.\

More recently, \citet{aghababaei2020quantifying} described a model for post-disaster business recovery that employs Bayesian methods. The main advantage of this approach, compared to approaches based on regression analysis, is that it can be constantly improved as new data becomes available; however, aggregating data from different communities and different disasters is not a trivial task.\

%--------------------------
\section{Major Research Gaps} 
%--------------------------
The recovery of businesses remains an under-studied topic, even when compared to the recovery of infrastructure and housing. Business recovery studies after catastrophic disasters has only been investigated for a handful of events, e.g., Hurricanes Ike and Katrina, and the Northridge and Canterbury earthquakes. It is unclear if the findings from these events are transferable to other locations and hazards. It is also unclear if business recovery after concentrated but destructive events shares the same patterns. \ 

Previous studies have been inconsistent in their findings. For example, the existence of a correlation between disaster preparedness and recovery performance is debatable. While a positive correlation is intuitively expected, studies of business recovery after major disasters show no relationship at all between preparedness measures and recovery outcomes \citet{webb2000businesses}. There remains the questions of whether or not these studies were designed in a way in which the effect of disaster preparedness on business recovery could be captured \citep{xiao2014hazard}. This is a topic that remains open and in need of further investigation.\ 

The role of entrepreneurial decision on the post-disaster of survival of business is another topic that deserves more attention. The majority of the studies conducted to date have focused on physical damage and disaster preparedness metrics. Nonetheless, it is argued that quick and informed decisions by the owners have had a large impact on business recovery \citep{morrish2020post}.\

