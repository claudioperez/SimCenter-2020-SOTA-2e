%%%%%%%%%%%%%%%%%%%% author.tex %%%%%%%%%%%%%%%%%%%%%%%%%%%%%%%%%%%
%
% sample root file for your "contribution" to a contributed volume
%
% Use this file as a template for your own input.
%
%%%%%%%%%%%%%%%% Springer %%%%%%%%%%%%%%%%%%%%%%%%%%%%%%%%%%%%%%%%%


%% RECOMMENDED %%%%%%%%%%%%%%%%%%%%%%%%%%%%%%%%%%%%%%%%%%%%%%%%%%%
%\documentclass[graybox]{svmult}
%
%% choose options for [] as required from the list
%% in the Reference Guide
%
%\usepackage{mathptmx}       % selects Times Roman as basic font
%\usepackage{helvet}         % selects Helvetica as sans-serif font
%\usepackage{courier}        % selects Courier as typewriter font
%\usepackage{type1cm}        % activate if the above 3 fonts are
                             % not available on your system
%
%\usepackage{makeidx}         % allows index generation
%\usepackage{graphicx}        % standard LaTeX graphics tool
%                             % when including figure files
%\usepackage{multicol}        % used for the two-column index
%\usepackage[bottom]{footmisc}% places footnotes at page bottom
%
%% see the list of further useful packages
%% in the Reference Guide
%
%\makeindex             % used for the subject index
%                       % please use the style svind.ist with
%                       % your makeindex program
%
%%%%%%%%%%%%%%%%%%%%%%%%%%%%%%%%%%%%%%%%%%%%%%%%%%%%%%%%%%%%%%%%%%%%%%%%%%%%%%%%%%%%%%%%%%
%
%\begin{document}

\title{Housing}
% Use \titlerunning{Short Title} for an abbreviated version of
% your contribution title if the original one is too long
\author{
    \textbf{Rodrigo Costa}
    \and{Ann-Margaret Esnard}
    \and{Henry Burton}}
\tocauthor{}
\authorrunning{Costa et al.}
% Use \authorrunning{Short Title} for an abbreviated version of
% your contribution title if the original one is too long
%\institute{Name of First Author \at Name, Address of Institute, %\email{name@email.address}
%\and Name of Second Author \at Name, Address of Institute %\email{name@email.address}}
%
% Use the package "url.sty" to avoid
% problems with special characters
% used in your e-mail or web address
%
\maketitle

The recovery of residential buildings is essential in restoring a sense of normalcy post-event. Recent events have demonstrated that housing recovery estimates based solely on building damage underestimate reality \citep{comerio2006estimating}. Repairs to residential buildings took between two and ten years after earthquakes in 1989 in Loma Prieta \citep{comerio2006estimating}, 1994 in Northridge \citep{Olshansky2006}, 1995 in Kobe \citep{comerio2014disaster, Olshansky2006}, 2009 in L'Aquila \citep{di2017reconstruction1, di2017reconstruction2}, 2010 in Chile \citep{comerio2013housing}, and 2011 in East Japan \citep{ranghieri2014learning}. The repairs needed after the 2010 and 2011 Canterbury earthquakes are expected to extend beyond 2020 \citep{BankOfNZ2016}. During recovery, economic growth and quality of life are impacted, and socioeconomic inequities can be exacerbated \citep{peacock2014inequities, wang2015influencing, bolin1985disasters}.\

It is generally accepted that predictive models of housing recovery can provide valuable insights and serve as a platform for evaluating the benefits of different mitigation actions. There is growing agreement that housing recovery needs to be modeled in the context of the community, being influenced by infrastructural and socioeconomic factors, and constrained by the availability of resources \citep{lee2019quantitative, masoomi2018community, ellingwood2018performance, bilau2018practice, Sutley2017a, davidson2015integrating}. This section introduces key concepts that become important when modeling the recovery of a portfolio of residential buildings: limited resources, decisions of owners, and interdependency with other systems. Previous sections have discussed the estimation of repair times for single- and multi-family buildings. This section assumes the reader is familiar with the concepts of repair-time estimation. \ 

%--------------------------
\section{Input and Output Data} 
%--------------------------
The simulation of housing recovery starts with estimating the repair time for buildings of interest. Repair-time estimation is discussed in detail in Part \ref{part:Performance}. In short, it requires information about building locations, type, and occupancy, as well as hazard levels at the buildings sites. Repair time is a measure of the time needed for damage to be fixed. Repair time should be differentiated from recovery time, which encompasses both the repair time and the time needed to procure the resources required for repairs to begin. The time needed to procure resources depends on building characteristics, as well as the resources of the building owners themselves. A building owner may be less capable of procuring resources due to financial constraints. Thus, two buildings requiring equal repair time may have significantly different recovery times. Because socioeconomic factors may affect the recovery capacity of the building owners, associating socioeconomic data to individual buildings becomes a requirement for housing recovery simulations. Since these data are not directly available from the Census, an alternative has been the use of synthetically generated data. In this case, aerial data aggregated at the Census tract/block level is used to simulate the socioeconomic demographics of the owners of each building in the portfolio \citep{rosenheim2019integration}. \ 

Another aspect of housing recovery modeling is the availability of resources. Disasters may lead to a surge in the demand for materials or skilled workers, and a reduction in resources, such as construction materials due to supply chain disruptions. The resource constraints will determine how many buildings can be repaired simultaneously, introducing another source for extensions in the recovery time. The number of workers in the construction sector can be estimated from Census data, city reports, permit data, or housing databooks \citep{kang2018replicating, costa2020housing}. \

Obtaining detailed data on all aspects highlighted above may prove to be a challenge. The level of detail sought in the simulation outputs should determine the level of detail in the input data. The use of ``what-if'' simulations to investigate the importance of certain considerations can help identify which inputs need to be refined. For example, one can compare a scenario where resource scarcity is not a problem to a scenario where only a fraction of the buildings can be repaired at the same time. This comparison provides insights on the importance of collecting detailed data on the availability of resources.\ 

%--------------------------
\section{Concepts in Housing Recovery Modeling} 
%--------------------------
As previously discussed, housing recovery is as much a socioeconomic process as an engineering process, as evidenced by a wide range of approaches that integrate social, economic, and engineering considerations observed in the recent literature (\cite{nejat2020spatially, moradi2020recovus, costa2020housing, bilau2018practice, hamideh2018housing, Burton2018, DESaster}). These integrated approaches are motivated by the premise that the homeowners are rational in that they must have both the means and willingness to embark on repairs and recovery. The means are often represented by financial capital. The willingness refers to the decision to repair immediately after the shock, wait and see, or sell the building. Socioeconomic status, perception of value of reconstruction, the decision of neighbors, and the state of the community have been shown to influence the ability to fund and willingness to repair for homeowners \citep{Burton2018, moradi2020recovus, nejat2012agent, comerio2006estimating, chang2011identifying, boiser2011skills, bilau2015framework, hwang2015postdisaster, bothara2016challenges}. \ 

Statistical and simulation-based models have been developed for post-disaster housing recovery. The former type of model is based on observations of past events and the use of machine-learning procedures. \citet{nejat2012agent} presents a model for housing recovery that uses the Least Absolute Shrinkage and Selection Operator (LASSO). A hierarchical Bayesian geo-statistical model has recently been proposed by \citet{nejat2020spatially}. Although statistical models are simpler to implement and have lower computational costs when compared to simulation-based models, the scarcity of data on post-disaster housing recovery is a limiting factor for using this type of approach.\ 

Simulation-based approaches rely on models for the hazard, its consequence, and the post-disaster recovery itself. Hazard and consequence modeling are discussed in the previous parts in this report. Regarding the recovery modeling, Figure \ref{fig:HousingRecovery} shows a conceptual algorithm for housing recovery simulation. The algorithm assumes that the recovery of a portfolio of $N_B$ buildings is of interest. Four decisions are evaluated sequentially. Decision (1) guarantees that the state of each building is evaluated on each time step. Decision (2) checks for the occurrence of a shock on the current time step. If the shock has occurred, damage, loss, and the time needed to repair building \textit{i}, $T_{r,i}$, are assessed. Part \ref{part:Performance} in this report describes the methodology to assess the immediate impact. If there is no shock, the assessment of impacts is skipped. Decision (3) checks if building \textit{i} is still damaged, and if it returns ``No,'' the algorithm moves to building $i+1$. Decision (4) is the core of the algorithm, where the decision of the building owners, the availability of resources, and neighborhood conditions among many other factors are evaluated to determine if the building can be repaired. If willingness and capability to repair are identified, the repairs can advance. The conceptual algorithm in Figure \ref{fig:HousingRecovery} should be evaluated repeatedly to simulate the progress along the timeline of recovery.\ 

\begin{figure}[htb]
    \centering
    \includegraphics[width=0.5\textwidth, angle = 0]{Figures/HousingRecovery.pdf}
    \caption{Conceptual algorithm for housing recovery.}
    \label{fig:HousingRecovery}
\end{figure}

The framework in Figure \ref{fig:HousingRecovery} is conceptual. More sophisticated frameworks that incorporate the concepts outlined in the figure are presented in  \citet{Sutley2017a}, \citet{Burton2018}, and \citet{costa2020housing}. In particular, Decision (4) in Figure \ref{fig:HousingRecovery} is complex, and many individual models may be necessary to assess it. Table \ref{tab:FactorsHousingRecovery} compiles some of the factors identified as drivers of recovery in past events. These factors should be acknowledged in Decision (4). The literature reviews in \citet{Costa2019thesis} and \citet{moradi2020recovus} are the main sources for Table \ref{tab:FactorsHousingRecovery}. Note: the majority of the studies in Table \ref{tab:FactorsHousingRecovery} investigated recovery of single-family buildings from earthquakes. \ 


%\begin{singlespace}
\begin{table}[tbh]
    \centering
%    \scriptsize
    \caption{Drivers of recovery.}
    \begin{tabularx}{\textwidth}{|l|X|}
    \toprule
    Race & \cite{wu2004comparative, kamel2004residential, bullard2009race, fussell2010race, peacock2014inequities,  nejat2019anchors} \\ \midrule
    
    Income & \cite{bolin1983recovery, wu2004comparative, zhang2009planning, peacock2014inequities, bilau2015framework,  wang2015influencing, hamideh2018housing, nejat2018demographics,  nejat2019anchors} \\ \midrule
    
    Immigration status & \cite{kamel2004residential,Nabil2004} \\ \midrule
    
    Education & \cite{burton2015validation, nejat2018demographics,  nejat2019anchors} \\ \midrule
    
    Family structure & \cite{Nejat2016, nejat2018perceived,  nejat2018demographics} \\ \midrule 
    
    Age & \cite{ngo2001disasters, sanders2004chapter, henderson2010older, nejat2018demographics} \\ \midrule
    
    Gender & \cite{nejat2018demographics} \\ \midrule
    
    Employment & \cite{bolin1983recovery, wang2015influencing} \\ \midrule
    
    Household size & \cite{sadri2018role, nejat2019anchors} \\ \midrule
    
    Home ownership & \cite{wu2004comparative, kamel2004residential, Nabil2004, zhang2009planning, peacock2014inequities,  hamideh2018housing, mayer2020drivers} \\ \midrule
    
    Disaster experience & \cite{kick2011repetitive, binder2015rebuild, nejat2018perceived, sadri2018role, moradi2020recovus} \\ \midrule
    
    Disaster damage & \cite{myers2008social, fussell2010race,  peacock2014inequities, mcneil2015household, hamideh2018housing, sadri2018role, mayer2020drivers} \\ \midrule 
    
    Social capital & \cite{airriess2008church, aldrich2010fixing, aldrich2011power, aldrich2012building, li2010katrina, burton2015validation, sadri2018role} \\ \midrule 
    
    Place attachment & \cite{chamlee2009there, cutter2010disaster, kick2011repetitive, binder2015rebuild, mcneil2015household, reid2015making} \\ \midrule
    
    Neighbor's recovery & \cite{dacy1969economics, rust2006financial, nejat2012agent} \\ \midrule 
    
    Financing & \cite{wu2004comparative, kamel2004residential, Nabil2004, comerio2014disaster, BankOfNZ2016, Nejat2016} \\ \midrule
    
    Materials & \cite{comerio2006estimating, REDi, bilau2015framework} \\ \midrule
    
    Skilled workforce & \cite{comerio2006estimating, chang2011identifying, boiser2011skills, bilau2015framework, hwang2015postdisaster, bothara2016challenges, REDi} \\ \midrule 
    
    Infrastructure & \cite{Miles2011, comerio2014disaster, burton2015validation, nejat2019anchors} \\ \midrule
    
    Community assets & \cite{Miles2011, comerio2014disaster, burton2015validation, nejat2019anchors} \\

    \bottomrule

    \end{tabularx}
    \label{tab:FactorsHousingRecovery}
\end{table}
%\end{singlespace}

%\FloatBarrier
%--------------------------
\section{Major Research Gaps and Needs}
%--------------------------

One of the main challenges for the development housing recovery models is verification and validation. Empirical data on housing recovery collected longitudinally and systematically are still scarce, as catastrophic events are rare and data have been consistently collected only recently  for the purpose of informing simulations models. The development of testbeds, as well as devoting funds to longitudinal studies of post-disaster housing recoveries are important steps to close this gap.\ 

Although current housing recovery simulations have limited predictive power, they are valuable tools for comparing proposed disaster recovery strategies. Housing recovery simulations can be employed to inform the development of new policies, by identifying regions or socioeconomic groups less capable of recovering. This topic has not yet been explored extensively from a research perspective.\ 

Risk is not static in time. As cities grow, new technologies are developed, population behaviors evolve, and the assets exposed as well as the vulnerabilities change. Integrating risk dynamics models with housing recovery simulations can significantly improve the assessment of the benefits of housing recovery policies. This is another topic that can benefit from further research.\

% %--------------------------
% \section{Software and Scripts}
% %--------------------------
% This section briefly introduces the main software and scripts for housing recovery simulation available in the literature, published in papers, dissertations, thesis, or user manuals. Statistical models are not included here.\ 

% \paragraph{DESaster} is a Python library for discrete event simulation of disaster recovery. Currently, DESaster focuses on housing recovery. The steps needed to repair residential buildings are the events, and these include the restoration of water and power supply, inspections, financing, and construction \citep{DESaster}.\ 

% \paragraph{RecovUS}  is a spatial agent-based model that includes spatial locations of homes and community assets and captures interactions of households with their neighbors and perceived community assets. It also simulates the effect of recovery of community, including infrastructure, neighbors, and community assets, on households' recovery decisions \citep{moradi2020recovus}. 

% \paragraph{Rts}  is a framework of agent-based object-oriented models developed to study housing recovery \citep{costa2020housing}, and population displacements \citep{costa2020predicting}. The Rts framework includes models for water, power, and transportation infrastructure, resource suppliers, and buildings \citep{Costa2019thesis}. A sophisticated modeling approach for the simulation of the requests and the delivery of supplies is also implemented. \ 

% \paragraph{\cite{markhvida2020quantification}} developed a model to simulate housing recovery and its impact on household well-being. They simulated the impact of building code improvements, increased insurance take-up rates, and availability unemployment insurance on the well-being of households affected by an earthquake.\

% \paragraph{\cite{sutley2018interdisciplinary}} \\ present an interdisciplinary model for post-disaster housing recovery that employs system dynamics. Their model account for physical damage, social vulnerability, resource availability and allocation, as well as the effect of policies. This model is expansible and versatile, allowing for different levels of detail to be used.\ 

% \paragraph{\cite{hwang2015postdisaster}} \\ introduce a system dynamics model to understand overall recovery efforts from a holistic perspective. They investigated the effectiveness of governmental budget allocation plans, and the prioritization of the recovery of selected sectors after the 2011 Tohoku Earthquake. \

% \paragraph{\cite{diaz2015housing}} \\ describe a system-dynamics framework aimed at evaluating the effects of different plans for the allocation of construction materials on housing recovery. The proposed model brings insights into the types of the housing recovery and the material demands over time.\

% \paragraph{\cite{grinberger2014bouncing}} \\ developed an agent-based model to simulate the housing recovery after earthquakes. The model was applied to study the impact of land-use, sheltering, public service replacement policies on housing recovery after a hypothetical earthquake near Jerusalem.\ 

\FloatBarrier





