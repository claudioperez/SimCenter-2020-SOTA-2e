%%%%%%%%%%%%%%%%%%%% author.tex %%%%%%%%%%%%%%%%%%%%%%%%%%%%%%%%%%%
%
% sample root file for your "contribution" to a contributed volume
%
% Use this file as a template for your own input.
%
%%%%%%%%%%%%%%%% Springer %%%%%%%%%%%%%%%%%%%%%%%%%%%%%%%%%%%%%%%%%


%% RECOMMENDED %%%%%%%%%%%%%%%%%%%%%%%%%%%%%%%%%%%%%%%%%%%%%%%%%%%
%\documentclass[graybox]{svmult}
%
%% choose options for [] as required from the list
%% in the Reference Guide
%
%\usepackage{mathptmx}       % selects Times Roman as basic font
%\usepackage{helvet}         % selects Helvetica as sans-serif font
%\usepackage{courier}        % selects Courier as typewriter font
%\usepackage{type1cm}        % activate if the above 3 fonts are
                             % not available on your system
%
%\usepackage{makeidx}         % allows index generation
%\usepackage{graphicx}        % standard LaTeX graphics tool
%                             % when including figure files
%\usepackage{multicol}        % used for the two-column index
%\usepackage[bottom]{footmisc}% places footnotes at page bottom
%
%% see the list of further useful packages
%% in the Reference Guide
%
%\makeindex             % used for the subject index
%                       % please use the style svind.ist with
%                       % your makeindex program
%
%%%%%%%%%%%%%%%%%%%%%%%%%%%%%%%%%%%%%%%%%%%%%%%%%%%%%%%%%%%%%%%%%%%%%%%%%%%%%%%%%%%%%%%%%%
%
%\begin{document}

\title{Geotechnical Systems}
% Use \titlerunning{Short Title} for an abbreviated version of
% your contribution title if the original one is too long
\author{
    \textbf{Pedro Arduino},
    \newline
    with review comments and suggestions by Jonathan D. Bray, Brady Cox, Michael Gardner, Boris Jeremic, David McCallen, Chaofeng Wang
}
\tocauthor{}
\authorrunning{Arduino}
% Use \authorrunning{Short Title} for an abbreviated version of
% your contribution title if the original one is too long
%\institute{Name of First Author \at Name, Address of Institute, %\email{name@email.address}
%\and Name of Second Author \at Name, Address of Institute %\email{name@email.address}}
%
% Use the package "url.sty" to avoid
% problems with special characters
% used in your e-mail or web address
%
\maketitle
\label{chapter:res_geotech}

Problems in geotechnical earthquake engineering often involve complex geometry and boundary conditions. Materials comprising geotechnical media behave almost always in a nonlinear fashion. Moreover, soil is made of three phases, and interactions between these phases play an important role in the global response, making theories even more complicated. Interaction of structural foundations (e.g., bridges, abutments, or buildings) with the surrounding soil is also a major aspect to consider in geotechnical earthquake analysis and design. Natural material inhomogeneity caused by soil deposition, as well as human influence, contributes to the complexity of the problem. In addition, the dynamic nature of earthquakes and their effects can rarely be considered in simplified models while preserving all their important aspects.

To address these problems, numerical analysis have become the most viable method for design and research purposes. In describing the state of the art in numerical modeling in geotechnical earthquake engineering, it is necessary to discuss each one of the aforementioned aspects; i.e., numerical methods, coupled formulations, constitutive models, interface elements, boundary and initial conditions, and corresponding verification and validation efforts. A list of common geotechnical codes used in geotechnical earthquake engineering is provided below. Although incomplete, this list identifies tools that address several common aspects. A few notes on research gaps and needs is included at the end to complete this section. 

\section{Numerical Methods}
% I used numbers for these section labels to save time because I expect that the structure of this chapter might change significantly. Eventually, I suggest using more descriptive labels by replacing the numbers with text.
\label{sec:resp_geotech_1}

Among many other methods of analysis, the finite-element method (FEM), finite-difference method (FDM), the material-point method (MPM), smooth particle hydrodynamics (SPH), and discrete-element method (DEM) are used in geotechnical earthquake engineering. Of these, the FEM and FDM are most common in geotechnical practice and research. Commercial codes such as \citeprgm{PLAXIS}, \citeprgm{FLAC}, \citeprgm{LS-DYNA}, and \citeprgm{ABAQUS} and open-source codes such as \citeprgm{OpenSees}, \citeprgm{FEAP}, and \citeprgm{Real-ESSI} are examples of numerical frameworks that offer dedicated geotechnical capabilities for earthquake applications. For one-dimensional wave propagation analysis, equivalent linear methods continue to be a common choice, with ``shake-like'' tools, such as \citeprgm{ProShake}, \citeprgm{Strata}, and \citeprgm{DeepSoil} (EL), being popular in practice. Most FEM tools offer one-dimensional, two-dimensional, and three-dimensional capabilities. Finite-element formulations that reduce computational demand via mesh accuracy, effective assimilation of nonlinear constitutive models, or general efficiency \citep{McGann12, McGann15} are ideal in this context. Today, extensive research is devoted in establishing finite-element formulations for solid mechanics that are equally applicable to any arbitrarily-posed problem.

When considering problems dominated by large deformations, such as the case of debris flows or tailing-dam runouts, meshless techniques, such as MPM and SPH, provide the necessary functionality to account for these conditions \citep{Mast15}. MPM codes commonly used in research and practice include \citeprgm{CB-Geo-MPM}, \citeprgm{Claymore}, \citeprgm{Uintah}, and \citeprgm{Anura3D}.

In recent years the discrete-element method (DEM) and discontinuous deformation analysis method (DDA) have gained applicability in geotechnical earthquake engineering, in particular for understanding phenomena at micro- and meso-scales and homogenization to the macro-scale \citep{kawamoto2018}. Popular DEM and DDA tools include PFC, LIGGGHTS, and LS-DEM.   

\section{Coupled Fluid--Solid Formulations}
\label{sec:resp_geotech_2}

Geotechnical earthquake engineering requires the evaluation of total and effective stresses. Total stress analysis is based on conventional single-phase formulations. Effective stress analysis requires a method to account for the interaction between the pore fluids and soil skeleton in saturated or partially saturated soil. Various approaches derived from the early work of \citet{Biot41, Biot56, Biot62} and others, including \citet{Borja06} and \citet{Ehlers02}, have been developed and added to multiple numerical frameworks, each one adding fluid degrees-of-freedom to the system using different assumptions \citep{Arduino01}. Three primary numerical approaches are discussed in \citet{Zienk84}. These approaches include the $u-p-U$ formulation (which uses the full coupled system of equations developed for a saturated problem), the $u-U$ formulation (which assumes both media are incompressible), and the $u-p$ approach (which simplifies the system by assuming that the fluid acceleration can be neglected). The $u-p$ approach is most common in commercial codes such as \citeprgm{PLAXIS} and \citeprgm{FLAC}, and is also available in \citeprgm{OpenSees} and \citeprgm{Real-ESSI}. These formulations have also found application in MPM codes, although at this level it is important to completely separate the phases. Extensive research is currently ongoing in this field. 

\section{Treatment of Soil--Foundation Interfaces}
\label{sec:resp_geotech_3}

Interaction of structural components with the surrounding soil is of major concern in geotechnical earthquake engineering. This issue arises in many geotechnical problems whether related to retaining structures, foundation engineering, underground construction, or even soil improvement systems, and is one of the most important and challenging aspects of geotechnical numerical modeling since it is inherently nonlinear and complex. Different approaches have been proposed over the past forty years that range from simple interaction springs ($p-y$, $t-z$,and $Q-z$ springs) \citep{API07} to methods based on contact mechanics (thin layer and interface elements) \citep[see][]{Laursen02}. Simplified models rely heavily on empirical methods. Extrapolating these methods to more complicated and general cases requires extreme scrutiny of the problem-at-hand and method used. The more advanced the methods are, the more complex and costly they become in terms of computation. \citeprgm{PLAXIS}, \citeprgm{FLAC}, \citeprgm{ABAQUS}, and \citeprgm{Real-ESSI} include interface elements, and \citeprgm{OpenSees} offers a suite of elements, including nonlinear springs, interface elements, and contact elements to characterize beam--solid interaction \citep[see][]{Petek06, Ghofrani18}. Coupling between structural systems and geotechnical domains rely on the appropriateness of these elements. Continued efforts and developments are underway in this field. 

\section{Soil Constitutive Modeling}
\label{sec:resp_geotech_4}

Constitutive models play a vital role in geotechnical earthquake engineering. Accurate and reliable numerical analyses require constitutive models to represent the \textit{in situ} soil response and different drainage and loading conditions. Over the years, soil constitutive models  have ranged from relatively simple von Mises, Drucker-Prager, and Mohr-Coulomb plasticity models to sophisticated models that capture specific aspects of soil behavior; \cite[see][]{Borja2013}. Among them Cam-Clay and other critical-state-based plasticity models have been of particular value and significance in geotechnical engineering. Most of these models use isotropic hardening and are applicable to static and monotonic loading conditions. Most geotechnical FE codes (PLAXIS, OpenSees, Real-ESSI, etc.) include conventional implementations of these models. For dynamic analysis, however, kinematic hardening is required to capture the cyclic nature of the soil response. For this purpose, three families of models are commonly used: multi-yield surface models, bounding surface models, and multiple-strain mechanisms models. These models differ in the way kinematic hardening is treated. Multi-surface plasticity models were introduced for soils by \citet{Prevost77, Prevost85a} and were extended to liquefiable sands by \citet{Elgamal03}. These models have been used to represent the constitutive behavior of both cohesive and cohesionless soils in total and effective stress analyses and are available in OpenSees. Bounding surface models were first introduced in geotechnical engineering by \citet{Dafalias86} and coworkers, and extended using critical-state concepts by \citet{Dafalias04} to represent the response of liquefiable soils. This model is available in OpenSees, Real-ESSI, and FLAC. Variations of this model by \citet{boulanger2017pm4sand, boulanger2018pm4silt} (referred to as PM4Sand and PM4Silt) have been developed to better represent the undrained cyclic response of sands and silts. These models are available in FLAC, PLAXIS, and OpenSees. The multi-mechanisms approach is defined in strain space and has been used in Japan, most particularly through its implementation in the Cocktail model proposed by \citet{iai2011dilatancy, iai2013finite}.

\section{Boundary and Loading Conditions}
\label{sec:resp_geotech_5}

Boundary conditions in geotechnical problems require special attention to ensure appropriate results. At a minimum, boundaries must be fixed such that all rigid-body displacement modes are restricted. In static and pseudo-static analyses, the main concern is diminishing the effects of the boundary on the portions of the model that are of primary interest. For the analysis of a soil--foundation system, boundary effects can be controlled by extending the limits of the soil domain away from the location of the foundation elements. Minimizing boundary effects is also critical in dynamic analysis; however, devising proper boundary conditions is more difficult than for static or pseudo-static cases. The particular method used for this purpose depends upon the objective of the numerical model and originates from the fact that the assumption of a rigid boundary is typically not valid. Several strategies have been proposed to accommodate the effect of a semi-infinite subsurface in a numerical model of finite size. The use of periodic boundary conditions, in which the lateral extents of the model share translational degrees-of-freedom, is one such approach that attempts to appropriately account for the free-field response of the soil domain. 

\citet{Lysmer69} introduced a technique to capture a transmitting boundary through the use of viscous dashpots. By defining the viscous response of the dashpots based on the density and shear-wave velocity of the material beyond the boundary, this approach appropriately captures the outward propagation of wave energy in the numerical model as long as the waves impinge in a near-normal orientation to the boundary. When defining transmitting boundaries using the \citet{Lysmer69} method, accelerations are not directly applied to the model. Instead, an effective force is applied using the technique developed by \citet{Joyner75}. This effective force is proportional to the input velocity and the constitutive properties of the material beyond the boundary. This approach is commonly used in numerical analysis for geotechnical problems to account for the compliance between the soil domain of the model and the semi-infinite media outside of the considered domain.

Better results can be attained using a perfectly matched layer (PML), which is an artificial absorbing layer for wave equations commonly used to truncate computational regions in numerical methods to simulate problems with open boundaries, especially in finite-difference and finite-element methods \citep{Zhang19}. PML's are designed so that waves incident upon the PML do not reflect back to the medium at the interface. In general, PMLs have been shown to produce better results than LK dashpots.

Lastly, a technique to properly account for the differences in wave behavior inside the finite soil domain represented by the model and the wave behavior in the semi-infinite soil medium is the domain reduction method \citep{Bielak03, Yoshimura03}. The domain reduction method (DRM) consists of two phases. The initial phase involves a background geological model that includes both the source of the earthquake and the region of interest. This background model is used to compute the free-field displacement wave-field demands on the boundary of the smaller region of interest. The second phase involves only the reduced region of interest. In this phase, effective seismic forces are applied at the boundary of the local region. These effective forces are derived from the boundary displacement demand obtained in the initial phase. In general, these methods require coupling data from different codes or accessing databases with recorded or synthetic motions. This is of particular importance in geotechnical earthquake engineering. The propagation of waves in geologic media can be simulated using codes such as broad band platforms (BBP) based on Green's functions and stochastic analysis. In general, these codes cannot represent the extreme soil nonlinearity observed at the surface where finite-element methods are more appropriate. When the response of a basin is of interest, finite-element and finite-difference codes, like Hercules or SW4, can be used to simulate the propagation of waves in large heterogeneous geologic domains, but they require extensive high-performance computing (HPC) resources to run properly. Independent of the tool used, coupling between these codes and conventional finite-element analysis is required. Efforts are underway to facilitate these simulations by the NHERI SimCenter and DesignSafe.

\section{Initial Conditions}
\label{sec:resp_geotech_6}

Representation of the initial state of stress and initial stress history is of paramount importance in geotechnical simulations. The soil response (i.e., stress--strain) greatly depends on initial conditions. Several approaches can be used to create an appropriate initial state. A typical method is to apply gravitational body forces to the elements prior to any static or dynamic analysis steps. Most tools facilitate this step by using a staged modeling procedure in which gravitational stresses are first developed in a base soil mesh. The stress history of the the soil (i.e., its overconsolidation ratio) should also be specified so the soil constitutive model responds correctly when first loaded. After this stage is complete, soil elements can be removed or added and replaced by foundation or additional soil elements, and gravitational stresses are developed for the new configuration.

\section{Verification and Validation} 
\label{sec:resp_geotech_7}

Over the years, as numerical formulations have become more refined they have also become more elaborate, adding complexity to their implementation and use. Therefore, before a newly proposed tool or model is used in practice or research, verification and validation (\emph{V\&V}) processes are necessary. \emph{Verification} is meant to identify and remove programming errors in computer codes and verify numerical algorithms. \emph{Validation} is meant to assess the accuracy at which a numerical model represents reality and includes the essential features of a real model. In contemporary numerical modeling, \emph{V\&V} has become an integral part of software development. Today, all numerical tools undergo exhaustive and continuous \emph{V\&V} processes. Recent comprehensive \emph{V\&V} efforts in geotechnical earthquake engineering include Prenolin, LEAP, and the NGL project. 

\section{Research Gaps and Needs}
\label{sec:resp_geotech_8}

\begin{itemize}

\item Development and implementation of advanced constitutive models for geotechnical earthquake engineering applications continue to be a challenge. In addition to model formulation and functionality, robustness and implementation efficiency are of paramount importance;  

\item Formulations capable of representing multi-phase materials including mixing and separation and large deformations continue to limit the applicability of numerical tools to simplified scenarios and conditions. Recent developments in FEM and meshless techniques are promising, although extensive work is needed;

\item Performing adequate soil--structure interaction (SSI) for major facilities continues to be a challenge and a gap in geotechnical earthquake engineering. In general, older codes (circa 1970s) continue to be the preferred option. Although very useful, these codes are based on simplifying assumptions, and do not offer real HPC capabilities; engineers are continually forced towards model-size-reduction compromises;

\item It is still common to idealize incident ground motions as pure vertically propagating shear and compression waves: this is not correct. Although the research community has engaged time-domain nonlinear SSI, there is still a lot of work to do to develop appropriate modeling solutions for regional simulations. This is really important for major infrastructure systems;

\item Integration of capabilities to execute regional-scale simulations of hazard and risk continues to be a challenge. Software descriptions for earthquake simulations are pretty much stand-alone and do not discuss the end-to-end coupling needed for regional simulations. It is important to begin discussions on approaches, challenges, and gaps for effective regional-scale simulation. This should include computational workflow strategies for handling massive amounts of data (both input and output) and rigorous coupling of geophysics and structural models; and

\item One of the important advancements in earthquake simulations has been the expansive availability of parallel platforms to the community. As a result, our ability to represent large geotechnical domains subject to static and dynamic loading conditions is improving at a rapid pace. In the particular case of geotechnical earthquake engineering, our capacity to compute ground motions to even higher frequencies is ever increasing. A challenge will be how to address geologic model uncertainties, i.e., how to select optimal subsurface models.

\end{itemize}

\section{Software and Systems}
\label{sec:response_geotech_tools}

The following list includes software mentioned in this section and commonly used in geotechnical earthquake engineering applications:

\paragraph{OpenSees}
The Open System for Earthquake Engineering Simulation (\citeprgm{OpenSees}) is an open-source software framework capable of performing fully nonlinear dynamic effective stress analyses \citep{mckenna2011opensees}. OpenSees is maintained by the Pacific Earthquake Engineering Research (PEER) Center and actively developed by researchers at UC Berkeley and various research institutions. Several commonly used soil constitutive models have been implemented in OpenSees, and additional models can be added based on user needs. The framework is capable of running on HPC systems and supports MacOS, Linux, and Windows operating systems (see \url{https://opensees.berkeley.edu/}).

\paragraph{FLAC}
Fast Lagrangian Analysis of Continua (\citeprgm{FLAC}), developed by the Itasca Consulting Group, is a proprietary finite-difference based software package capable of performing dynamic nonlinear effective stress analyses. FLAC allows users to import custom soil constitutive models either as pre-compiled dynamic libraries or by using the scripting language FISH. FLAC is not capable of executing on HPC systems and is closed source. Currently only Windows-based operating systems are supported (see \url{https://www.itascacg.com/software/FLAC}).

\paragraph{PLAXIS}
\citeprgm{PLAXIS}, now part of Bentley Systems, is a finite-element software package that can be used to perform dynamic nonlinear effective stress analyses. Custom soil constitutive models can be implemented in within the platform. PLAXIS is proprietary and closed source. Currently, it is not HPC capable and supports only Windows-based operating systems (see \url{https://www.bentley.com/en/products/brands/plaxis}).

\paragraph{FEAP} The Finite Element Analysis Program (\citeprgm{FEAP}) is a general-purpose finite-element program for solving nonlinear, static, and transient partial differential equations. Its primary applications are directed to the solution of problems in solid mechanics; however, the system may be extended to solve problems in other subject areas by adding user developed modules to address problems in fluid dynamics, flow through porous media, thermo-electric fields, and others. The software is available to run on UNIX/Linux/Mac or Windows environments (see \url{http://feap.berkeley.edu/}).

\paragraph{Real-ESSI} The \citeprgm{Real-ESSI} Simulator ( Realistic Modeling and Simulation of Earthquakes, Soils, and Structures and their Interaction) is a finite-element framework for high-performance (sequential or parallel) analysis of linear and nonlinear geotechnical and structural systems including soil structure interaction. This software, developed at the University of California, Davis, is available to run on UNIX/Linux/Mac or Windows environments (see \url{http://real-essi.us/})

\paragraph{General FEM solvers}
\citeprgm{LS-DYNA} and \citeprgm{ABAQUS} are  proprietary general finite-element method solvers capable of fully nonlinear dynamic effective stress analyses. Custom material models, such as those required for modeling dynamic soil response, can be implemented in these frameworks. Depending on the license purchased, LS-Dyna and ABAQUS are capable of running on HPC systems. LS-Dyna supports Unix, Linux, and Windows-based operating systems, and is currently available on DesignSafe. ABAQUS currently supports Linux and Windows-based operating systems.

\paragraph{General 1D tools}
\citeprgm{Strata}, \citeprgm{ProShake}, and \citeprgm{DeepSoil} are well establish tools to perform one-dimensional linear-elastic and equivalent-linear (SHAKE-type) site response analyses. Deepsoil also includes nonlinear capabilities and pore-water pressure generation and dissipation models.

\paragraph{General MPM solvers}
In recent years, several MPM codes have been proposed. Among them the \citeprgm{CB-Geo-MPM}, \citeprgm{Claymore}, \citeprgm{Uintah}, and \citeprgm{Anura3D} are well established open-source MPM frameworks. CB-Geo MPM is a highly parallel distributed material point method for solving large-deformation problems in geotechnical engineering. The code supports isoparametric elements, multiphase materials, and photorealistic rendering (see 
\url{https://github.com/cb-geo/mpm}). Claymore is a massively parallel and scalable multi-GPU MPM framework for simulating physical behaviors of materials undergoing complex topological changes, self-collision, and large deformations (see \url{https://github.com/penn-graphics-research/claymore}). The Uintah software suite is a set of libraries and applications for solving partial differential equations on structured adaptive grids using hundreds to thousands of processors. MPM functionality has been added to the code and used for simulating granular flow, high-impact metal deformation, and fluid-structure interaction (see \url{http://uintah.utah.edu/}). The Anura3D Software is a three-dimensional implementation of the material point method developed and used alongside field and laboratory experiments for understanding the physics involved in soil--water--structure interaction (see \url{http://www.mpm-dredge.eu/}).

\paragraph{General DEM solvers}
Several DEM codes are available. Among them \citeprgm{PFC}, \citeprgm{LIGGGHTS}, and LS-DEM have been used in geotechnical and geological settings.
\citeprgm{PFC} is a general discrete-element method (DEM) framework developed by the Itasca Consulting group. Together with FLAC, this tool is commonly used for the analysis of geotechnical and geological problems.
LIGGGHTS, an open-source (at least partially) DEM package, is capable of performing both pseudo-static and dynamic analyses. Some functionality within LIGGGHTS is not available in the public version (see \url{https://github.com/CFDEMproject/LIGGGHTS-PUBLIC}). LS-DEM is a variant of DEM that uses level set functions to represent the shapes of constituent particles \citep{kawamoto2018}. When used to represent a particle, a level set function is an implicit function whose value at a given point is the signed distance from that point to the surface of the particle. In a DEM framework, this formulation is convenient because two of the most important ingredients in DEM contact detection---interparticle penetration distance and contact normal---are given. Using this methodology, the tool is able to handle real, complex particle morphologies, which enable reproduction and prediction of the bulk behavior of experimental specimens.

\subsection{Relevant SimCenter Tools}

The SimCenter develops both research and educational tools to facilitate performing geotechnical engineering simulations. The tools currently available focus on the analysis of individual sites and help with one-dimensional propagation of ground shaking from the bedrock to the free surface and calibration of numerical materials and models used in geotechnical earthquake engineering. These methods will be added to \citeprgm{R2DTool} to enable the sophisticated simulation of soil behavior in regional-scale analyses.

\paragraph{quoFEM} The Quantified Uncertainty with Optimization for the Finite-Element Method (\emph{quoFEM}) tool facilitates model calibration, optimization, and sensitivity analyses of numerical materials, components, and systems by combining existing simulation environments with state-of-the-art uncertainty quantification applications. The graphical user interface currently supports finite-element software (\citeprgm{OpenSees} and \citeprgm{FEAP}) and can also interface with custom analysis packages, including, but not limited to those based on the discrete element and finite difference method and other commercial software that cannot be bundled with the open-source SimCenter application (e.g., \citeprgm{LS-DYNA}, \citeprgm{ABAQUS}).

\paragraph{QS3HARK} The Site-Specific Seismic Hazard Analysis and Research Kit with Uncertainty Quantification (\emph{QS3HARK}) performs site-specific analysis of ground shaking and liquefaction by simulating wave propagation through soil layers. The simulations use the finite element method as implemented in \citeprgm{OpenSees} to perform the calculations. Several advanced material models are available in QS3HARK (e.g., PM4Sand, PM4Silt, PDMY, PDMY02, PDMY03, ManzariDafalias, Borja-Amies) to support complex site-response analysis. Uncertainties in both the soil properties and the bedrock ground motion inputs can be characterized and propagated throughout the simulations to arrive at a probabilistic description of the ground shaking.

\paragraph{EE-UQ and PBE} The Earthquake Engineering with Uncertainty Quantification (\citeprgm{EE-UQ}) and the Performance-Based Engineering (\citeprgm{PBE}) tools facilitate the assessment of structural response and the expected damage and losses from earthquake events. QS3HARK is integrated in these tools to allow for propagation of ground shaking from the bedrock to the surface and apply the simulated surface ground motions to a building model. This allows sophisticated consideration of soil behavior for the performance assessment of buildings. The workflow used in these tools for individual buildings will be integrated into \citeprgm{R2DTool}, the regional simulation tool of the SimCenter to enable such high-fidelity analyses at the regional scale.

\paragraph{Educational Tools} The Site-Specific Seismic Hazard Analysis and Research Kit (\citeprgm{S3HARK}) is the educational version of QS3HARK that provides the same features and versatile set of material models but without uncertainty quantification. Removing UQ supports educational needs by streamlining the user interface and facilitating the setup of simulations.

\noindent The Pile Group Tool (\citeprgm{PGT}) supports studies on the behavior of a pile or a pile-group in layered soil under ground shaking. Its user interface allows students to interactively observe the response of the system to changes in the soil characteristics, the pile configuration, and the pile design.

\noindent The Transfer Function Tool (\citeprgm{TFT}) focuses on calculating the transfer function that maps a given bedrock motion to a surface motion. The tool allows students to define the characteristics of layered soil profiles and examine how the input motion is affected by the soil deposit.
