%%%%%%%%%%%%%%%%%%%%%%preface.tex%%%%%%%%%%%%%%%%%%%%%%%%%%%%%%%%%%%%%%%%%
% sample preface
%
% Use this file as a template for your own input.
%
%%%%%%%%%%%%%%%%%%%%%%%% Springer %%%%%%%%%%%%%%%%%%%%%%%%%%

\preface

This report is a product of the NHERI SimCenter under the auspices of the U.S. National Science Foundation (NSF). It provides an overview and review of simulation requirements and software tools for natural hazards engineering (NHE) of the built environment. The simulations discussed in this report are an essential component of research to address the three grand challenge areas and associated research questions outlined in the NHERI Science Plan \citep{edge2020natural}. These grand challenges entail: (1) identifying and quantifying the characteristics natural hazards that are damaging to civil infrastructure and disruptive to communities; (2) evaluating the physical vulnerability of civil infrastructure and the social vulnerability of populations in at-risk communities; and (3) creation of technologies and tools to design, retrofit, and operate a resilient and sustainable infrastructure for the Nation. Accordingly, required simulation technologies encompass a broad range of phenomena and considerations, from characterization and simulation of natural hazards and their damaging effects on buildings and civil infrastructure, to quantifying the resulting economic losses, disruption, and other consequences on society. Ultimately, the goal is to enable high-fidelity and high-resolution models in regional simulations that can support technological, economic, and policy solutions to mitigate the threat of natural hazards.

The natural hazards addressed in this report include earthquakes, tsunami, storm and tornado winds, and storm surge. While not an exhaustive list of all possible natural hazards, these are the hazards addressed under NSF's NHERI research program. The first chapter of the report provides an introduction to the SimCenter and its goals, including an overview of the plans and status for software tool development.  The subsequent chapters of the report are organized into five parts in a sequential fashion, including: (1) simulation methods to characterize the natural hazards; (2) response simulation of structural and geotechnical systems and localized wind and water flows; (3) quantifying the resulting damage and its effects on the performance of buildings, transportation systems, and utility infrastructure systems; (4) strategies and emerging tools to model recovery from natural disasters; and (5) the cross-cutting applications of uncertainty quantification methods and artificial intelligence to NHE. 

Owing to the broad scope of the simulation topics, this state-of-art review is presented with the goal of educating and informing researchers---including both simulation tool developers and users---on key requirements and capabilities within each simulation topic. The report is also a guide to the on-going development of simulation capabilities by the NSF NHERI SimCenter. Each chapter of the report begins with a brief overview of the purpose of the simulation component, including a discussion of the goals of the analysis (what is being calculated), the underlying physics or principles involved in the simulation, common modeling assumptions and simplifications, and typical input and output of the simulations. 

With the aim of taking stock of computational simulation capabilities, informing the NHERI community of research advances to date, and positioning the work of the NHERI SimCenter as it relates to computational simulation, the summaries identify and review commonly used simulation software that is widely known and used for research in academia and industry. Particular emphasis is placed on open-source or other software that is hosted on DesignSafe or is otherwise easily accessible to researchers; a summary table of the simulation software tools is provided as an appendix to the report. In addition to summarizing the state-of-art in the various topic areas, each chapter of the report identifies major research gaps and needs, with the intent that these could motivate research proposals to NSF or other agencies that will lead to future advancements.

This report is an update to a State-of-the Art Report that the SimCenter first published in February 2019.  This update reflects comments and suggestions that were solicited from leading researchers in NHE; it includes new chapters on disaster recovery modeling and applications of artificial intelligence technologies to NHE.  Readers are encouraged to contribute feedback regarding this report and the SimCenter simulation tool development through the online SimCenter Forum at http://simcenter-messageboard.designsafe-ci.org/smf/.


\vspace{\baselineskip}
\begin{flushright}\noindent
Stanford University,\hfill {\it Gregory G. Deierlein}\\
December, 2020\hfill {\it Adam Zsarnóczay}\\
\end{flushright}


