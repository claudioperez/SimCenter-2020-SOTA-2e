%%%%%%%%%%%%%%%%%%%%% appendix.tex %%%%%%%%%%%%%%%%%%%%%%%%%%%%%%%%%
%
% sample appendix
%
% Use this file as a template for your own input.
%
%%%%%%%%%%%%%%%%%%%%%%%% Springer-Verlag %%%%%%%%%%%%%%%%%%%%%%%%%%

\chapter{List of Software Tools}

The tables in this appendix were prepared to provide an overview of a few important features of the tools mentioned in each chapter. The presented information is meant to help researchers compare the available tools and see which one fits their needs. 


The type of license and supported platforms are based on the available descriptions and data on each software's website. An empty license means that the developers have not provided that information; it does not necessarily mean that the software is freeware. Even though applications with an open source code could often be compiled on all three platforms considered here, the information in the tables corresponds to their availability for typical usage. That is, if there are pre-built binaries available and typical users would download those binaries, then the information in the tables refers to the platforms supported by those binaries.


Ticks in the DesignSafe column identify applications that are either available in the Workspace on the DesignSafe website or they can take advantage of DesignSafe resources by establishing a secure remote connection to the service. 


The identification of certain commercial systems and research tools in the tables does not imply recommendation or endorsement by the SimCenter or the academic institutions of contributing authors. Nor does such identification imply that those products are necessarily the best available for the task.


The abbreviations used in the tables are described below.

...


% This file was generated using the Python script `scripts/make-tables.py`,
% which can be invoked by running `make tables` at the command line.

% Claudio M. Perez

\begin{center}


\begin{table}[]
    \caption{Introduction}
    \begin{threeparttable}
    \centering
    %\begin{tabular}{l|cccc}
    \begin{tabular}{p{3cm}|cccc}
    \toprule
    Name &  License & Operating system & DesignSafe & Notes \\ 
    \href{https://github.com/NHERI-SimCenter/WE-UQ}{WE-UQ} & - &  Mac/ Win/ Lin &  \checkmark &  \\
    \href{https://github.com/NHERI-SimCenter/EE-UQ}{EE-UQ} &  BSD  &  Mac/ Win/ Lin &  \checkmark &  \\
    \href{https://simcenter.designsafe-ci.org/research-tools/pbe-application/}{PBE} &  BSD  &  Mac/ Win/ Lin &  \checkmark & \tnotex{2} \\
    \href{https://nheri-simcenter.github.io/pelicun/}{PELICUN} &  BSD  &  Mac/ Win/ Lin & - &  \\
    \href{https://openfoam.org.}{OpenFOAM} & - & - & - &  \\
    \href{https://opensees.berkeley.edu/}{OpenSees} &  BSD  &  Mac/ Win/ Lin &  \checkmark & \tnotex{1} \\
    \href{https://depts.washington.edu/clawpack/geoclaw/}{GEOCLAW} & - &  Lin &  \checkmark &  \\
    \href{https://nheri-simcenter.github.io/BRAILS-Documentation/}{BRAILS} & - &  Mac/ Win/ Lin & - &  \\
    \href{https://github.com/NHERI-SimCenter/HydroUQ}{HydroUQ} & - &  Mac/ Win/ Lin & - &  \\
    \href{https://github.com/NHERI-SimCenter/quoFEM}{quoFEM} &  BSD  &  Mac/ Win/ Lin &  \checkmark &  \\ 
    \bottomrule
   %\insertTableNotes
    \end{tabular}

    \begin{tablenotes}\footnotesize
      \item[1]{ Finite-Element based}
\item[2]{ Building level}
    \end{tablenotes}
    \end{threeparttable}
    \label{tab:app-0}
\end{table}
\newline
\vspace*{1 cm}
\newline

\begin{table}[]
    \caption{Hazards: Earthquake - Ground Shaking}
    \begin{threeparttable}
    \centering
    %\begin{tabular}{l|cccc}
    \begin{tabular}{p{3cm}|cccc}
    \toprule
    Name &  License & Operating system & DesignSafe & Notes \\ 
    \href{https://zenodo.org/record/1063644#.X5PqWe1lBhE}{SW4} & LGPLv2 &  Lin & - & \tnotex{3} \\
    \href{https://strike.scec.org/scecpedia/Broadband_Platform}{The Broadband Platform} &  Apache Software  &  Lin & - &  \\
    Hazus 4.2 & - & - & - & \tnotex{1},\tnotex{2} \\
    \href{https://github.com/usgs/nshmp-haz/wiki}{NSHMP-Haz} & - &  Mac/ Win/ Lin & - &  \\
    \href{https://opensha.org/}{OpenSHA} &  Apache Software  &  Mac/ Win/ Lin & - &  \\
    \href{http://hpgeoc.github.io/awp-odc-os/}{AWP-ODC} &  BSD  &  Lin & - &  \\
    \href{https://www.globalquakemodel.org/openquake}{OpenQuake} &  &  Mac/ Win/ Lin &  ? & \tnotex{2} \\
    \href{http://www.r-crisis.com/}{R-CRISIS} & - & - & - &  \\
    UGMS MCER & - & - & - &  \\
    \href{https://scec.usc.edu/scecpedia/CyberShake}{CyberShake} &  BSD  &  Lin & - &  \\
    \href{https://github.com/abrahamson/HAZ}{HAZ} & GPLv3 &  Win & - &  \\ 
    \bottomrule
   %\insertTableNotes
    \end{tabular}

    \begin{tablenotes}\footnotesize
      \item[1]{ HAZUS MH only}
\item[2]{ Regional level}
\item[3]{ Finite-Difference based}
    \end{tablenotes}
    \end{threeparttable}
    \label{tab:app-1}
\end{table}
\newline
\vspace*{1 cm}
\newline

\begin{table}[]
    \caption{Hazards: Earthquake - Surface Fault Rupture}
    \begin{threeparttable}
    \centering
    %\begin{tabular}{l|cccc}
    \begin{tabular}{p{3cm}|cccc}
    \toprule
    Name &  License & Operating system & DesignSafe & Notes \\ 
    \href{http://mimetics-engineering.fr/index.php/en/lmgc90-2/}{LBMC90} & - & - & - &  \\
    \href{https://github.com/lammps}{LAMMPS} & GPL & - & - & \tnotex{3} \\
    \href{https://yade-dem.org/doc/}{YADE} & - & - & - &  \\
    \href{http://www.lstc.com/}{LS-DYNA} &  Other/Proprietary  &  Win/ Lin &  \checkmark & \tnotex{1} \\
    \href{https://www.itascainternational.com/software/pfc}{PFC} &  Other/Proprietary  &  Win & - & \tnotex{2} \\
    \href{https://www.bentley.com/en/products/brands/plaxis}{PLAXIS} &  Other/Proprietary  &  Win & - & \tnotex{1} \\
    \href{https://www.itascacg.com/software/FLAC}{FLAC} &  Other/Proprietary  &  Win & - & \tnotex{4} \\
    \href{https://www.cfdem.com/}{LIGGGHTS-PUBLIC} & GPLv2+ &  Lin & - & \tnotex{2} \\
    \href{www.simulia.com}{ABAQUS} &  Other/Proprietary  &  Win/ Lin &  \checkmark & \tnotex{1} \\
    \href{https://opensees.berkeley.edu/}{OpenSees} &  BSD  &  Mac/ Win/ Lin &  \checkmark & \tnotex{1} \\ 
    \bottomrule
   %\insertTableNotes
    \end{tabular}

    \begin{tablenotes}\footnotesize
      \item[1]{ Finite-Element based}
\item[2]{ Pseudostatic & dynamic analysis}
\item[3]{ Comment a}
\item[4]{ Finite-Difference based}
    \end{tablenotes}
    \end{threeparttable}
    \label{tab:app-2}
\end{table}
\newline
\vspace*{1 cm}
\newline

\begin{table}[]
    \caption{Hazards: Earthquake - Soil Liquefaction}
    \begin{threeparttable}
    \centering
    %\begin{tabular}{l|cccc}
    \begin{tabular}{p{3cm}|cccc}
    \toprule
    Name &  License & Operating system & DesignSafe & Notes \\ 
    \href{https://www.itascacg.com/software/FLAC}{FLAC} &  Other/Proprietary  &  Win & - & \tnotex{3} \\
    Liquefy-Pro &  Other/Proprietary  &  Win & - & \tnotex{1} \\
    \href{https://pm4sand.engr.ucdavis.edu/}{PM4Sand} & - & - & - &  \\
    NovoLIQ &  Other/Proprietary  &  Win & - & \tnotex{1} \\
    \href{https://pm4silt.engr.ucdavis.edu/}{PM4Silt} & - & - & - &  \\
    Cliq &  Other/Proprietary  &  Win & - & \tnotex{1} \\
    \href{https://www.bentley.com/en/products/brands/plaxis}{PLAXIS} &  Other/Proprietary  &  Win & - & \tnotex{2} \\
    LiqIT &  Other/Proprietary  &  Win & - & \tnotex{1} \\ 
    \bottomrule
   %\insertTableNotes
    \end{tabular}

    \begin{tablenotes}\footnotesize
      \item[1]{ Simplified methods}
\item[2]{ Finite-Element based}
\item[3]{ Finite-Difference based}
    \end{tablenotes}
    \end{threeparttable}
    \label{tab:app-3}
\end{table}
\newline
\vspace*{1 cm}
\newline

\begin{table}[]
    \caption{Hazards: Earthquake - Slope Stability and Landslides}
    \begin{threeparttable}
    \centering
    %\begin{tabular}{l|cccc}
    \begin{tabular}{p{3cm}|cccc}
    \toprule
    Name &  License & Operating system & DesignSafe & Notes \\ 
    \href{http://www.lstc.com/}{LS-DYNA} &  Other/Proprietary  &  Win/ Lin &  \checkmark & \tnotex{2} \\
    \href{https://www.itascacg.com/software/FLAC}{FLAC} &  Other/Proprietary  &  Win & - & \tnotex{3} \\
    \href{www.simulia.com}{ABAQUS} &  Other/Proprietary  &  Win/ Lin &  \checkmark & \tnotex{2} \\
    SLAMMER &  &  Mac/ Win/ Lin & - & \tnotex{1} \\
    \href{https://www.bentley.com/en/products/brands/plaxis}{PLAXIS} &  Other/Proprietary  &  Win & - & \tnotex{2} \\
    \href{https://opensees.berkeley.edu/}{OpenSees} &  BSD  &  Mac/ Win/ Lin &  \checkmark & \tnotex{2} \\ 
    \bottomrule
   %\insertTableNotes
    \end{tabular}

    \begin{tablenotes}\footnotesize
      \item[1]{ Newmark sliding block}
\item[2]{ Finite-Element based}
\item[3]{ Finite-Difference based}
    \end{tablenotes}
    \end{threeparttable}
    \label{tab:app-4}
\end{table}
\newline
\vspace*{1 cm}
\newline

\begin{table}[]
    \caption{Hazards: Tropical Cyclone - Storm Surge}
    \begin{threeparttable}
    \centering
    %\begin{tabular}{l|cccc}
    \begin{tabular}{p{3cm}|cccc}
    \toprule
    Name &  License & Operating system & DesignSafe & Notes \\ 
    \href{https://oss.deltares.nl/web/delft3d}{Delft3D} & - & - & - &  \\
    \href{https://adcirc.org/}{ADCIRC} &  NA &  Mac/ Win/ Lin &  \checkmark &  \\
    \href{https://depts.washington.edu/clawpack/geoclaw/}{GEOCLAW} & - &  Lin &  \checkmark &  \\
    \href{https://www.clawpack.org/}{Clawpack} & - & - & - &  \\
    \href{http://fvcom.smast.umassd.edu/fvcom/}{FVCOM} & - & - & - &  \\
    SLOSH &  NA &  NA & - & \tnotex{1} \\ 
    \bottomrule
   %\insertTableNotes
    \end{tabular}

    \begin{tablenotes}\footnotesize
      \item[1]{ Surge hazard results available from NOAA}
    \end{tablenotes}
    \end{threeparttable}
    \label{tab:app-6}
\end{table}
\newline
\vspace*{1 cm}
\newline

\begin{table}[]
    \caption{Response: Structural Systems}
    \begin{threeparttable}
    \centering
    %\begin{tabular}{l|cccc}
    \begin{tabular}{p{3cm}|cccc}
    \toprule
    Name &  License & Operating system & DesignSafe & Notes \\ 
    \href{http://feap.berkeley.edu/}{FEAP} &  Free For Educational Use & - & - & \tnotex{1} \\
    \href{http://www.lstc.com/}{LS-DYNA} &  Other/Proprietary  &  Win/ Lin &  \checkmark & \tnotex{1} \\
    \href{https://opensees.berkeley.edu/}{OpenSees} &  BSD  &  Mac/ Win/ Lin &  \checkmark & \tnotex{1} \\ 
    \bottomrule
   %\insertTableNotes
    \end{tabular}

    \begin{tablenotes}\footnotesize
      \item[1]{ Finite-Element based}
    \end{tablenotes}
    \end{threeparttable}
    \label{tab:app-8}
\end{table}
\newline
\vspace*{1 cm}
\newline

\begin{table}[]
    \caption{Response: Geotechnical Systems}
    \begin{threeparttable}
    \centering
    %\begin{tabular}{l|cccc}
    \begin{tabular}{p{3cm}|cccc}
    \toprule
    Name &  License & Operating system & DesignSafe & Notes \\ 
    \href{http://www.lstc.com/}{LS-DYNA} &  Other/Proprietary  &  Win/ Lin &  \checkmark & \tnotex{1} \\
    \href{http://real-essi.us/}{Real-ESSI} & - & - & - &  \\
    \href{https://github.com/arkottke/strata}{Strata} & - & - & - &  \\
    \href{https://pm4sand.engr.ucdavis.edu/}{PM4Sand} & - & - & - &  \\
    \href{https://pm4silt.engr.ucdavis.edu/}{PM4Silt} & - & - & - &  \\
    \href{https://www.itascacg.com/software/FLAC}{FLAC} &  Other/Proprietary  &  Win & - & \tnotex{2} \\
    \href{www.simulia.com}{ABAQUS} &  Other/Proprietary  &  Win/ Lin &  \checkmark & \tnotex{1} \\
    \href{http://www.mpm-dredge.eu/}{Anura3D} &  Other/Proprietary  &  Win & - & \tnotex{4} \\
    \href{http://uintah.utah.edu/}{Uintah} &  MIT  &  Lin & - & \tnotex{4} \\
    \href{https://github.com/penn-graphics-research/claymore}{Claymore} & - & - & - &  \\
    \href{http://feap.berkeley.edu/}{FEAP} &  Free For Educational Use & - & - & \tnotex{1} \\
    \href{https://www.bentley.com/en/products/brands/plaxis}{PLAXIS} &  Other/Proprietary  &  Win & - & \tnotex{1} \\
    \href{http://www.proshake.com/}{ProShake} &  Other/Proprietary  &  Win & - & \tnotex{2},\tnotex{3} \\
    \href{http://deepsoil.cee.illinois.edu/}{DeepSoil} &  Other/Proprietary  &  Win &  \checkmark & \tnotex{2},\tnotex{3} \\
    \href{https://www.itascainternational.com/software/pfc}{PFC} &  Other/Proprietary  &  Win & - & \tnotex{5} \\
    \href{https://www.cb-geo.com/research/mpm/}{CB-Geo-MPM} &  MIT  & - & - &  \\
    \href{https://opensees.berkeley.edu/}{OpenSees} &  BSD  &  Mac/ Win/ Lin &  \checkmark & \tnotex{1} \\ 
    \bottomrule
   %\insertTableNotes
    \end{tabular}

    \begin{tablenotes}\footnotesize
      \item[1]{ Finite-Element based}
\item[2]{ Finite-Difference based}
\item[3]{ Transfer function}
\item[4]{ Material point method}
\item[5]{ Pseudostatic & dynamic analysis}
    \end{tablenotes}
    \end{threeparttable}
    \label{tab:app-9}
\end{table}
\newline
\vspace*{1 cm}
\newline

\begin{table}[]
    \caption{Response: Computational Fluid Dynamics - Wind}
    \begin{threeparttable}
    \centering
    %\begin{tabular}{l|cccc}
    \begin{tabular}{p{3cm}|cccc}
    \toprule
    Name &  License & Operating system & DesignSafe & Notes \\ 
    \href{https://doi.org/10.5281/zenodo.3462804}{Turbulent Inflow Tool} & - & - & - &  \\ 
    \bottomrule
   %\insertTableNotes
    \end{tabular}

    \begin{tablenotes}\footnotesize
      
    \end{tablenotes}
    \end{threeparttable}
    \label{tab:app-10}
\end{table}
\newline
\vspace*{1 cm}
\newline

\begin{table}[]
    \caption{Response: Computational Fluid Dynamics - Water}
    \begin{threeparttable}
    \centering
    %\begin{tabular}{l|cccc}
    \begin{tabular}{p{3cm}|cccc}
    \toprule
    Name &  License & Operating system & DesignSafe & Notes \\ 
    \href{https://www.comsol.com.}{COMSOL} & - & - & - &  \\
    \href{olaflow.github.io}{olaFlow} & - & - & - &  \\
    \href{https://engys.com/.}{H.E.L.Y.X.} & - & - & - &  \\
    \href{https://openfoam.org.}{OpenFOAM} & - & - & - &  \\
    \href{https://su2code.github.io/.}{SU2} & LGPLv2 &  Mac/ Win/ Lin & - &  \\
    \href{https://ihfoam.ihcantabria.com/.}{IHFOAM} & - & - & - &  \\
    \href{https://www.ansys.com/products/fluids/ansys-fluent.}{ANSYS Fluent} &  Other/Proprietary  & - & - &  \\
    \href{https://mdx.plm.automation.siemens.com/star-ccm-plus.}{S.T.A.R.-C.C.M.+} & - & - & - &  \\ 
    \bottomrule
   %\insertTableNotes
    \end{tabular}

    \begin{tablenotes}\footnotesize
      
    \end{tablenotes}
    \end{threeparttable}
    \label{tab:app-11}
\end{table}
\newline
\vspace*{1 cm}
\newline

\begin{table}[]
    \caption{Performance: Buildings}
    \begin{threeparttable}
    \centering
    %\begin{tabular}{l|cccc}
    \begin{tabular}{p{3cm}|cccc}
    \toprule
    Name &  License & Operating system & DesignSafe & Notes \\ 
    Hazus 4.2 & - & - & - & \tnotex{5},\tnotex{4} \\
    \href{https://github.com/openslat/SLAT}{OpenSLAT} & GPLv3 & - & - &  \\
    \href{https://simcenter.designsafe-ci.org/research-tools/pbe-application/}{PBE} &  BSD  &  Mac/ Win/ Lin &  \checkmark & \tnotex{1} \\
    \href{https://nheri-simcenter.github.io/pelicun/}{PELICUN} &  BSD  &  Mac/ Win/ Lin & - &  \\
    \href{https://femap58.atcouncil.org/pact}{PACT} & - &  Win & - & \tnotex{1},\tnotex{2} \\
    \href{https://ecapra.org/}{CAPRA} & - &  Win & - & \tnotex{3},\tnotex{4} \\
    \href{https://www.hbrisk.com/}{SP3} &  Other/Proprietary  & - & - & \tnotex{1},\tnotex{2} \\
    \href{http://mae.cee.illinois.edu/software/software_maeviz.html}{MAEViz} & - & - & - &  \\
    \href{https://www.globalquakemodel.org/openquake}{OpenQuake} &  &  Mac/ Win/ Lin &  ? & \tnotex{4} \\ 
    \bottomrule
   %\insertTableNotes
    \end{tabular}

    \begin{tablenotes}\footnotesize
      \item[1]{ Building level}
\item[2]{ FEMA P58 only}
\item[3]{ HAZUS based}
\item[4]{ Regional level}
\item[5]{ HAZUS MH only}
    \end{tablenotes}
    \end{threeparttable}
    \label{tab:app-12}
\end{table}
\newline
\vspace*{1 cm}
\newline

\begin{table}[]
    \caption{Performance: Transportation Networks}
    \begin{threeparttable}
    \centering
    %\begin{tabular}{l|cccc}
    \begin{tabular}{p{3cm}|cccc}
    \toprule
    Name &  License & Operating system & DesignSafe & Notes \\ 
    Hazus 4.2 & - & - & - & \tnotex{1},\tnotex{2} \\
    \href{https://www.globalquakemodel.org/openquake}{OpenQuake} &  &  Mac/ Win/ Lin &  ? & \tnotex{2} \\ 
    \bottomrule
   %\insertTableNotes
    \end{tabular}

    \begin{tablenotes}\footnotesize
      \item[1]{ HAZUS MH only}
\item[2]{ Regional level}
    \end{tablenotes}
    \end{threeparttable}
    \label{tab:app-13}
\end{table}
\newline
\vspace*{1 cm}
\newline

\begin{table}[]
    \caption{Performance: Water, Sewer, and Gas Pipelines}
    \begin{threeparttable}
    \centering
    %\begin{tabular}{l|cccc}
    \begin{tabular}{p{3cm}|cccc}
    \toprule
    Name &  License & Operating system & DesignSafe & Notes \\ 
    GIRAFFE & - &  Win & - & \tnotex{1} \\
    \href{https://www.epa.gov/water-research/epanet}{EPANET} &  Public Domain & - & - & \tnotex{1} \\
    WNTR & - & - & - & \tnotex{1} \\ 
    \bottomrule
   %\insertTableNotes
    \end{tabular}

    \begin{tablenotes}\footnotesize
      \item[1]{ Potable water network}
    \end{tablenotes}
    \end{threeparttable}
    \label{tab:app-14}
\end{table}
\newline
\vspace*{1 cm}
\newline

\begin{table}[]
    \caption{Performance: Electrical Transmission Substations and Lines}
    \begin{threeparttable}
    \centering
    %\begin{tabular}{l|cccc}
    \begin{tabular}{p{3cm}|cccc}
    \toprule
    Name &  License & Operating system & DesignSafe & Notes \\ 
    Hazus 4.2 & - & - & - & \tnotex{1},\tnotex{2} \\
    \href{https://matpower.org/}{MatPower} &  BSD  & - & - &  \\
    \href{https://sourceforge.net/p/electricdss/wiki/Home/}{OpenDSS} &  BSD  & - & - & \tnotex{3} \\ 
    \bottomrule
   %\insertTableNotes
    \end{tabular}

    \begin{tablenotes}\footnotesize
      \item[1]{ HAZUS MH only}
\item[2]{ Regional level}
\item[3]{ Electrical networks}
    \end{tablenotes}
    \end{threeparttable}
    \label{tab:app-15}
\end{table}
\newline
\vspace*{1 cm}
\newline

\begin{table}[]
    \caption{Recovery: Communities}
    \begin{threeparttable}
    \centering
    %\begin{tabular}{l|cccc}
    \begin{tabular}{p{3cm}|cccc}
    \toprule
    Name &  License & Operating system & DesignSafe & Notes \\ 
    DESaster & - & - & - &  \\
    \href{http://resilience.colostate.edu/in_core.shtml}{IN-CORE} & - & - & - &  \\ 
    \bottomrule
   %\insertTableNotes
    \end{tabular}

    \begin{tablenotes}\footnotesize
      
    \end{tablenotes}
    \end{threeparttable}
    \label{tab:app-16}
\end{table}
\newline
\vspace*{1 cm}
\newline

\begin{table}[]
    \caption{Recovery: Housing}
    \begin{threeparttable}
    \centering
    %\begin{tabular}{l|cccc}
    \begin{tabular}{p{3cm}|cccc}
    \toprule
    Name &  License & Operating system & DesignSafe & Notes \\ 
    DESaster & - & - & - &  \\ 
    \bottomrule
   %\insertTableNotes
    \end{tabular}

    \begin{tablenotes}\footnotesize
      
    \end{tablenotes}
    \end{threeparttable}
    \label{tab:app-18}
\end{table}
\newline
\vspace*{1 cm}
\newline

\begin{table}[]
    \caption{Cross-Cutting: Uncertainty Quantification}
    \begin{threeparttable}
    \centering
    %\begin{tabular}{l|cccc}
    \begin{tabular}{p{3cm}|cccc}
    \toprule
    Name &  License & Operating system & DesignSafe & Notes \\ 
    \href{http://www.cossan.co.uk/software/cossan-x.php}{COSSAN-X} & - & - & - &  \\
    \href{http://www.cossan.co.uk/software/open-cossan-engine.php}{OpenCOSSAN} & LGPLv3 & - & - &  \\
    \href{https://dakota.sandia.gov/}{Dakota} & LGPL &  Mac/ Win/ Lin &  \checkmark &  \\
    \href{https://www.uqlab.com/}{UQLab} &  Free For Educational Use & - & - &  \\
    \href{http://muq.mit.edu/}{MUQ} & - & - & - &  \\ 
    \bottomrule
   %\insertTableNotes
    \end{tabular}

    \begin{tablenotes}\footnotesize
      
    \end{tablenotes}
    \end{threeparttable}
    \label{tab:app-20}
\end{table}
\newline
\vspace*{1 cm}
\newline

\begin{table}[]
    \caption{Cross-Cutting: Artificial Intelligence and Machine Learning}
    \begin{threeparttable}
    \centering
    %\begin{tabular}{l|cccc}
    \begin{tabular}{p{3cm}|cccc}
    \toprule
    Name &  License & Operating system & DesignSafe & Notes \\ 
    \href{https://caffe.berkeleyvision.org/}{CAFFE} & - & - & - &  \\
    \href{https://pytorch.org/}{PyTorch} & - &  Mac/ Win/ Lin & - &  \\
    \href{http://deeplearning.net/software/theano/}{Theano} & - & - & - &  \\
    \href{https://keras.io/}{Keras} & - &  Mac/ Win/ Lin & - &  \\
    \href{https://www.tensorflow.org/}{TensorFlow} &  Apache Software  & - & - &  \\
    \href{https://docs.microsoft.com/en-us/cognitive-toolkit/}{CNTK} &  MIT  & - & - &  \\
    \href{https://scikit-learn.org}{Scikit-Learn} & - & - & - &  \\ 
    \bottomrule
   %\insertTableNotes
    \end{tabular}

    \begin{tablenotes}\footnotesize
      
    \end{tablenotes}
    \end{threeparttable}
    \label{tab:app-21}
\end{table}
\newline
\vspace*{1 cm}
\newline
\end{center}

